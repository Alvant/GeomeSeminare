\documentclass[a4paper,12pt]{article}

\usepackage{mystyle}
\usepackage{gensymb}


\usepackage{scalerel}
\usepackage{stackengine}

\graphicspath{ {images/} }


\definecolor{violet}{RGB}{148, 0, 211}
\definecolor{red}{RGB}{183, 65, 14}



% https://tex.stackexchange.com/a/101138/135045

\newcommand\widesim[1]{\ThisStyle{%
  \setbox0=\hbox{$\SavedStyle#1$}%
  \stackengine{-.1\LMpt}{$\SavedStyle#1$}{%
    \stretchto{\scaleto{\SavedStyle\mkern.2mu\sim}{.5150\wd0}}{.6\ht0}%
  }{O}{c}{F}{T}{S}%
}}

\newcommand{\BigMiddleThree}{\;\left|\vphantom{\begin{pmatrix} 0\\0\\0 \end{pmatrix}}\right.\;}


\author{Алексеев Василий}


\title{Семинар 1}
\date{4 февраля + 5 февраля 2021}


\begin{document}
  \maketitle
  
  \tableofcontents

  \thispagestyle{empty}
  
  \newpage
  
  \pagenumbering{arabic}


  \section{Ранг матрицы}
  
  
  \section{Задачи}
  
  
  \subsection{\# 15.45(2)}
  
  Вычислить обратную для матрицы
  \[
    A = \begin{pmatrix}
      2 & -1 & 0\\
      0 & 2 & -1\\
      -1 & -1 & 1
    \end{pmatrix}
  \]
  
  \begin{solution}
    Найдём обратную с помощью метода Гаусса.
    Далее \textcolor{violet}{фиолетовым} цветом будем выделять элемент в столбце, с помощью которого будем занулять другие элементы в том же столбце.
    Те, которые зануляем на данном шаге, будем отмечать \textcolor{red}{красным} цветом.
    Когда столбец ``готов'' (остался один ненулевой~---~фиолетовый), переходим к другому столбцу и снова выбираем ненулевой элемент для ``зачищения столбца'', но \emph{из строчек, откуда ещё не выбирали}.
    
    \begin{equation*}
    \begin{split}
      &\left(
        \begin{matrix}
          2 & -1 & 0\\
          0 & 2 & -1\\
          -1 & -1 & 1
        \end{matrix}
        \BigMiddleThree
        \begin{matrix}
          1 & 0 & 0\\
          0 & 1 & 0\\
          0 & 0 & 1
        \end{matrix}
        \right)\\
      \widesim{(1) \leftrightarrow (3)}\quad &\left(
        \begin{matrix}
          \textcolor{violet}{\bds{-1}} & -1 & 1\\
          0 & 2 & -1\\
          \textcolor{red}{\bds 2} & -1 & 0
        \end{matrix}
        \BigMiddleThree
        \begin{matrix}
          1 & 0 & 0\\
          0 & 1 & 0\\
          1 & 0 & 0
        \end{matrix}
        \right)\\
      \widesim{(3) = (3) + (1) \cdot 2}\quad &\left(
        \begin{matrix}
          -1 & -1 & 1\\
          0 & 2 & -1\\
          0 & -3 & 2
        \end{matrix}
        \BigMiddleThree
        \begin{matrix}
          1 & 0 & 0\\
          0 & 1 & 0\\
          1 & 0 & 2
        \end{matrix}
        \right)\\
      \widesim{(2) = (2) / 2}\quad &\left(
        \begin{matrix}
          -1 & \textcolor{red}{\bds{-1}} & 1\\
          0 & \textcolor{violet}{\bds 1} & -1/2\\
          0 & \textcolor{red}{\bds{-3}} & 2
        \end{matrix}
        \BigMiddleThree
        \begin{matrix}
          1 & 0 & 0\\
          0 & 1/2 & 0\\
          1 & 0 & 2
        \end{matrix}
        \right)\\
      \widesim{\substack{(3) = (3) + (2) \cdot 3\\(1) = (1) + (2)}}\quad &\left(
        \begin{matrix}
          -1 & 0 & \textcolor{red}{\bds{1/2}}\\
          0 & 1 & \textcolor{red}{\bds{-1/2}}\\
          0 & 0 & \textcolor{violet}{\bds{1/2}}
        \end{matrix}
        \BigMiddleThree
        \begin{matrix}
          0 & 1/2 & 1\\
          0 & 1/2 & 0\\
          1 & 3/2 & 2
        \end{matrix}
        \right)\\
      \widesim{\substack{(2) = (2) + (3)\\(1) = (1) - (3)}}\quad &\left(
        \begin{matrix}
          -1 & 0 & 0\\
          0 & 1 & 0\\
          0 & 0 & 1/2
        \end{matrix}
        \BigMiddleThree
        \begin{matrix}
          -1 & -1 & -1\\
          1 & 2 & 2\\
          1 & 3/2 & 2
        \end{matrix}
        \right)\\
      \widesim{\substack{(1) = -1 \cdot (1)\\(3) = 2 \cdot (3)}}\quad &\left(
        \begin{matrix}
          1 & 0 & 0\\
          0 & 1 & 0\\
          0 & 0 & 1
        \end{matrix}
        \BigMiddleThree
        \begin{matrix}
          1 & 1 & 1\\
          1 & 2 & 2\\
          2 & 3 & 4
        \end{matrix}
        \right)
    \end{split}
    \end{equation*}
    
    Таким образом,
    \[
      A^{-1} = \begin{pmatrix}
        1 & 1 & 1\\
        1 & 2 & 2\\
        2 & 3 & 4
      \end{pmatrix}
    \]
    
    Можно проверить:
    \[
      AA^{-1} = \begin{pmatrix}
        2 & -1 & 0\\
        0 & 2 & -1\\
        -1 & -1 & 1
      \end{pmatrix}
      \cdot \begin{pmatrix}
        1 & 1 & 1\\
        1 & 2 & 2\\
        2 & 3 & 4
      \end{pmatrix}
      = \begin{pmatrix}
        1 & 0 & 0\\
        0 & 1 & 0\\
        0 & 0 & 1
      \end{pmatrix}
    \]
    
    Почему преобразования строк у ``сдвоенной'' матрицы позволило найти $A^{-1}$?
    Каждое элементарное преобразование строк\footnote{Умножение строки на число, отличное от нуля, и прибавление к одной строке другой} можно задать невырожденной матрицей $S$, умножение на которую слева равносильно данному преобразованию строк.
    Таким образом, каждый шаг метода Гаусса можно рассматривать как умножение слева на некоторую $S_i$:
    \[
      \bigl(A \mid E\bigr) \to \bigl(S_1 A \mid S_1 E\bigr) \to \ldots
      \to \bigl(\overbrace{S_n \ldots S_1 A}^{E} \mid \overbrace{S_n \ldots S_1 E}^{B}\bigr)
    \]
    где единичная матрица $E \hm= S_n \ldots S_1 A$~---~то, что стремимся получить слева, справа же получается матрица $B \hm= S_n \ldots S_1 E \hm= S_n \ldots S_1$.
    Выходит, $E \hm= BA$, что равносильно тому, что $B \hm= A^{-1}$.
    
    \bigskip
    
    \emph{Отступление}
    
    Найдём интереса ради какую-нибудь $S_i$.
    Например, $S_1$, которая задаёт перестановку строк.
    Правда, перестановка строк~---~не совсем элементарное преобразование.
    Разложим его сначала на элементарные.
    
    Мы хотим задать преобразование перестановки строк (первой и третьей):
    \[
      \begin{pmatrix}
        2 & -1 & 0\\
        0 & 2 & -1\\
        -1 & -1 & 1
      \end{pmatrix}
      \longrightarrow \begin{pmatrix}
        -1 & -1 & 1\\
        0 & 2 & -1\\
        2 & -1 & 0
      \end{pmatrix}
    \]
    
    Это преобразование можно представить как композицию преобразований (над-под каждой стрелочкой обозначено элементарное преобризование и его матрица\footnote{Матрица, которую можно получить,например, из единичной, проведя над её строками аналогичное преобразование.}):
    \begin{equation*}
    \begin{split}
      &\begin{pmatrix}
        2 & -1 & 0\\
        0 & 2 & -1\\
        -1 & -1 & 1
      \end{pmatrix}\\
      \xrightarrow[(3) = (3) + (1)]{\left(\begin{smallmatrix}1 & 0 & 0\\0 & 1 & 0\\1 & 0 & 1\end{smallmatrix}\right)}\quad &\begin{pmatrix}
          2 & -1 & 0\\
          0 & 2 & -1\\
          1 & -2 & 1
        \end{pmatrix}\\
      \xrightarrow[(1) = (1) - (3)]{\left(\begin{smallmatrix}1 & 0 & -1\\0 & 1 & 0\\0 & 0 & 1\end{smallmatrix}\right)}\quad &\begin{pmatrix}
          1 & 1 & -1\\
          0 & 2 & -1\\
          1 & -2 & 1
        \end{pmatrix}\\
      \xrightarrow[(3) = (3) + (1)]{\left(\begin{smallmatrix}1 & 0 & 0\\0 & 1 & 0\\1 & 0 & 1\end{smallmatrix}\right)}\quad &\begin{pmatrix}
          1 & 1 & -1\\
          0 & 2 & -1\\
          2 & -1 & 0
        \end{pmatrix}\\
      \xrightarrow[(1) = -1 \cdot (1)]{\left(\begin{smallmatrix}-1 & 0 & 0\\0 & 1 & 0\\0 & 0 & 1\end{smallmatrix}\right)}\quad &\begin{pmatrix}
          -1 & -1 & 1\\
          0 & 2 & -1\\
          2 & -1 & 0
        \end{pmatrix}
    \end{split}
    \end{equation*}
    
    И в итоге, $S_1$, задающая первую перестановку строк:
    \[
      S_1 = \begin{pmatrix}
          -1 & 0 & 0\\
          0 & 1 & 0\\
          0 & 0 & 1
        \end{pmatrix}
      \cdot \begin{pmatrix}
          1 & 0 & 0\\
          0 & 1 & 0\\
          1 & 0 & 1
        \end{pmatrix}
      \cdot \begin{pmatrix}
          1 & 0 & -1\\
          0 & 1 & 0\\
          0 & 0 & 1
        \end{pmatrix}
      \cdot \begin{pmatrix}
          1 & 0 & 0\\
          0 & 1 & 0\\
          1 & 0 & 1
        \end{pmatrix}
      = \ldots
    \]
  \end{solution}
  
  
  \subsection{\# 15.65(1)}
  
  \[
    \begin{pmatrix}
      2 & 5\\
      1 & 3
    \end{pmatrix} X = \begin{pmatrix}
      2 & 1\\
      1 & 1
    \end{pmatrix}
  \]
  
  \begin{solution}
    Какая размерность у матрицы $X$?
    Очевидно, если $X_{m\times n}$, то $m \hm= 2$ и $n \hm= 2$.
    
    Также можно заметить, что матрица, которая известный множитель, и матрица-результат невырождены.
    Поэтому и $X$ должна быть невырождена.
    
    Чтобы найти $X$, можно умножить обе части уравнение на обратную к матрице-множителю:
    \[
    \begin{pmatrix}
      2 & 5\\
      1 & 3
    \end{pmatrix}^{-1}
    \begin{pmatrix}
      2 & 5\\
      1 & 3
    \end{pmatrix} X = \begin{pmatrix}
      2 & 5\\
      1 & 3
    \end{pmatrix}^{-1}
    \begin{pmatrix}
      2 & 1\\
      1 & 1
    \end{pmatrix}
  \]
  
  В итоге (после нахождения обратной по формуле или с помощью метода Гаусса) получается, что
  \[
    X \hm= \begin{pmatrix}
      1 & -2\\
      0 & 1
    \end{pmatrix}
  \]
  \end{solution}
  
  \bigskip
  
  \emph{Отступление}
  
  Рассмотрим уравнение в форме
  \[
    AX = B
  \]
  
  В задаче $A$ и $B$ были невырождены.
  Но что было бы, если бы одна или две из них были бы вырождены?
  
  Уравнение
  \[
    \mbox{(невыр.)}\ X = \mbox{(выр.)}
  \]
  очевидно, не имеет решений.
  
  То же самое можно сказать и про уравнение
  \[
    \mbox{(выр.)}\ X = \mbox{(невыр.)}
  \]
  
  Но в случае
  \[
    \mbox{(выр.)}\ X = \mbox{(выр.)}
  \]
  всё зависит от конкретных матриц.
  
  Например, пусть дано уравнение
  \[
    \begin{pmatrix}
      1 & 1\\
      1 & 1
    \end{pmatrix} X = \begin{pmatrix}
      1 & 3\\
      2 & 6
    \end{pmatrix}
  \]
  
  Если искать матрицу $X$ в виде $X \hm= \left(\begin{smallmatrix}a & b\\ c & d\end{smallmatrix}\right)$, то матричное уравнение можно будет переписать как систему их четырёх скалярных уравнений
  \[
    \left\{
      \begin{aligned}
        &a + c = 1\\
        &b + d = 3\\
        &a + c = 2\\
        &b + d = 6
      \end{aligned}
    \right.
  \]
  у которой нет решений.
  
  Но для уравнения же
  \[
    \begin{pmatrix}
      1 & 2\\
      1 & 2
    \end{pmatrix} X = \begin{pmatrix}
      2 & 1\\
      2 & 1
    \end{pmatrix}
  \]
  аналогичная система скалярных уравнений
  \[
    \left\{
      \begin{aligned}
        &a + 2c = 2\\
        &b + 2d = 1\\
        &a + 2c = 2\\
        &b + 2d = 1
      \end{aligned}
    \right.
  \]
  уже будет иметь решение, причём не одно\footnote{Дело в том, что умножение матрицы $A$ справа на матрицу $X$ равносильно преобразованию столбцов (которое задаёт матрица $X$) матрицы $A$. Для рассматриваемого уравнения $AX \hm= B$ такое преобразование столбцов, очевидно, существует.}. И решение $X$ уравнение в общем виде будет
  \[
    X \hm= \begin{pmatrix}
      2 - 2c & 1 - 2d\\
      c & d
    \end{pmatrix},\quad c, d \in \RR
  \]
  ($c$ и $d$ выбраны в качестве параметрических переменных, остальные две из системы выражены через них)
  
  
  \subsection{\# 16.19(3)}
  
  \[
    A = \begin{pmatrix}
      1 & 1 & 1\\
      1 & \alpha & \alpha\\
      1 & \alpha^2 & \alpha^2
    \end{pmatrix}
  \]
  \[
    \Rg A = \?
  \]
  
  \begin{solution}
    Чтобы найти ранг матрицы, можно методом Гаусса привести её к упрощённому виду (то есть ``пытаться'' преобразованиями строк привести матрицу $A$ к единичной; некоторые строки в общем случае могут оказаться нулевыми).
    Первым преобразованием можно вычесть первую строку из второй и третьей:
    \[
      \begin{pmatrix}
        1 & 1 & 1\\
        1 & \alpha & \alpha\\
        1 & \alpha^2 & \alpha^2
      \end{pmatrix}
      \longrightarrow
      \begin{pmatrix}
      1 & 1 & 1\\
      0 & \alpha - 1 & \alpha - 1\\
      0 & \alpha^2 - 1 & \alpha^2 - 1
    \end{pmatrix}
    \]
    
    Далее метод Гаусса продолжать нельзя, пока не стало понятно, нулевые вторая и третья строки или нет.
    Например, следующим шагом метода Гаусса могла бы быть ``чистка второго столбца'', но для этого во второй или третьей строчке должен быть ненулевой элемент во втором столбце.
    \[
      \alpha - 1 \overset{\?}{=} 0 \Rightarrow \alpha = 1 \Rightarrow A = \begin{pmatrix}
        1 & 1 & 1\\
        0 & 0 & 0\\
        0 & 0 & 0
      \end{pmatrix} \Rightarrow \Rg A = 1
    \]
    \[
      \alpha^2 - 1 \overset{\?}{=} 0 \Rightarrow \alpha = -1\ \mbox{($1$ уже рассмотрели)} \Rightarrow A = \begin{pmatrix}
        1 & 1 & 1\\
        0 & -2 & -2\\
        0 & 0 & 0
      \end{pmatrix} \Rightarrow \Rg A = 2
    \]
    
    Остаются случаи $\alpha \hm{\not=} 1; 2$.
    При таком $\alpha$ вторая и третья строки точно ненулевые.
    Если они к тому же линейно независимы, то ранг матрицы $A$ будет равен $3$.
    Если линейно зависимы~---~ранг будет равен $2$.
    Найдём, какие $\alpha$ будут давать ранг $2$:
    \[
      \alpha - 1 = k \cdot (\alpha^2 - 1),\quad k \in \RR \setminus \{0\}
    \]
    В результате получаем $\alpha \hm= \frac{1}{k} \hm- 1$.
    То есть ранг матрицы $A$ при всех оставшихся $\alpha$ будет равен $2$.
  \end{solution}
\end{document}
