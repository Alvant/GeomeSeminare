\documentclass[a4paper,12pt]{article}

\usepackage{mystyle}
\usepackage{gensymb}


\usepackage{scalerel}
\usepackage{stackengine}

\graphicspath{ {images/} }


% https://tex.stackexchange.com/a/101138/135045

\newcommand\widesim[1]{\ThisStyle{%
  \setbox0=\hbox{$\SavedStyle#1$}%
  \stackengine{-.1\LMpt}{$\SavedStyle#1$}{%
    \stretchto{\scaleto{\SavedStyle\mkern.2mu\sim}{.5150\wd0}}{.6\ht0}%
  }{O}{c}{F}{T}{S}%
}}

\newcommand{\BigMiddleThree}{\;\left|\vphantom{\begin{pmatrix} 0\\0\\0 \end{pmatrix}}\right.\;}


\author{Алексеев Василий}


\title{Семинар 2}
\date{11 февраля + 12 февраля 2020}


\begin{document}
  \maketitle
  
  \tableofcontents

  \thispagestyle{empty}
  
  \newpage
  
  \pagenumbering{arabic}


  \section{Системы линейных уравнений}
  
  \section{Задачи}
  
  
  \subsection{\# 17.1(4)}
  
  Выписать расширенную матрицу.
  Решить систему уравнений:
  \[
    \left\{
    \begin{aligned}
      &y + 3z = -1\\
      &2x + 3y + 5z = 3\\
      &3x + 5y + 7z = 6
    \end{aligned}
    \right.
  \]
  
  \begin{solution}
    \begin{equation*}
    \begin{split}
      &\left(
          \begin{matrix}
            0 & 1 & 3\\
            2 & 3 & 5\\
            3 & 5 & 7
          \end{matrix}
          \BigMiddleThree
          \begin{matrix}
            -1\\
            3\\
            6
          \end{matrix}
        \right)\\
      \xrightarrow{\substack{(2) = (2) - 3 \cdot (1)\\(3) = (3) - 5 \cdot (1)}}\quad &\left(
          \begin{matrix}
            0 & 1 & 3\\
            2 & 0 & -4\\
            3 & 0 & -8
          \end{matrix}
          \BigMiddleThree
          \begin{matrix}
            -1\\
            6\\
            11
          \end{matrix}
        \right)\\
      \xrightarrow{(2) = (2) / 2}\quad &\left(
          \begin{matrix}
            0 & 1 & 3\\
            1 & 0 & -2\\
            3 & 0 & -8
          \end{matrix}
          \BigMiddleThree
          \begin{matrix}
            -1\\
            3\\
            11
          \end{matrix}
        \right)\\
      \xrightarrow{(3) = (3) - 3 \cdot (2)}\quad &\left(
          \begin{matrix}
            0 & 1 & 3\\
            1 & 0 & -2\\
            0 & 0 & -2
          \end{matrix}
          \BigMiddleThree
          \begin{matrix}
            -1\\
            3\\
            2
          \end{matrix}
        \right)\\
      \xrightarrow{(3) = -1/2 \cdot (3)}\quad &\left(
          \begin{matrix}
            0 & 1 & 3\\
            1 & 0 & -2\\
            0 & 0 & 1
          \end{matrix}
          \BigMiddleThree
          \begin{matrix}
            -1\\
            3\\
            -1
          \end{matrix}
        \right)\\
      \xrightarrow{\substack{(1) = (1) - 3 \cdot (3)\\(2) = (2) + 2 \cdot (3)}}\quad &\left(
          \begin{matrix}
            0 & 1 & 0\\
            1 & 0 & 0\\
            0 & 0 & 1
          \end{matrix}
          \BigMiddleThree
          \begin{matrix}
            2\\
            1\\
            -1
          \end{matrix}
        \right)\\
    \end{split}
    \end{equation*}
    
    Упрощённой матрице соответствует система
    \[
      \left\{
        \begin{aligned}
          &\hphantom{x +} y \hphantom{+ z} = 2\\
          &x \hphantom{+ y} \hphantom{+ z} = 1\\
          &\hphantom{x +} \hphantom{y +} z = -1\\
        \end{aligned}
      \right.
    \]
    решение которой, очевидно, $(2, 1, -1)^T$.
  \end{solution}
  
  
  \subsection{\# 19.6(20)}
  
  Решить систему с расширенной матрицей
  \[
    \left(
      \begin{matrix}
        1 & 2 & -7 & -3\\
        -5 & 4 & 63 & 29\\
        5 & 24 & -7 & -1
      \end{matrix}
      \BigMiddleThree
      \begin{matrix}
        -3\\
        71\\
        41
      \end{matrix}
    \right)
  \]
  
  \begin{solution}
    Расширенной матрице соответсвует система:
    \[
      \left\{
        \begin{aligned}
          &x_1 + 2x_2 - 7x_3 - 3x_4 = -3\\
          &-5x_1 + 4x_2 + 63x_3 + 29x_4 = 71\\
          &5x_1 + 24x_2 - 7x_3 - x_4 = 41
        \end{aligned}
      \right.
    \]
    
    Для её решения снова воспользуемся методом Гаусса приведения расширенной матрицы к упрощённому виду:
    
    \begin{equation*}
    \begin{split}
      &\left(
          \begin{matrix}
            1 & 2 & -7 & -3\\
            -5 & 4 & 63 & 29\\
            5 & 24 & -7 & -1
          \end{matrix}
          \BigMiddleThree
          \begin{matrix}
            -3\\
            71\\
            41
          \end{matrix}
        \right)\\
      \xrightarrow{\substack{(2) = (2) + 5 \cdot (1)\\(3) = (3) - 5 \cdot (1)}}\quad &\left(
          \begin{matrix}
            1 & 2 & -7 & -3\\
            0 & 14 & 28 & 14\\
            0 & 14 & 28 & 14
          \end{matrix}
          \BigMiddleThree
          \begin{matrix}
            -3\\
            56\\
            56
          \end{matrix}
        \right)\\
      \xrightarrow{\substack{(3) = (3) - (2)\\(2) = (2) / 14}}\quad &\left(
          \begin{matrix}
            1 & 2 & -7 & -3\\
            0 & 1 & 2 & 1\\
            0 & 0 & 0 & 0
          \end{matrix}
          \BigMiddleThree
          \begin{matrix}
            -3\\
            4\\
            0
          \end{matrix}
        \right)\\
      \xrightarrow{(1) = (1) - 2 \cdot (2)}\quad &\left(
          \begin{matrix}
            1 & 0 & -11 & -5\\
            0 & 1 & 2 & 1\\
            0 & 0 & 0 & 0
          \end{matrix}
          \BigMiddleThree
          \begin{matrix}
            -11\\
            4\\
            0
          \end{matrix}
        \right)
    \end{split}
    \end{equation*}
    
    Упрощённой матрице соответствует система
    \[
      \left\{
        \begin{aligned}
          &x_1 \hphantom{+ x_2} - 11x_3 - 5x_4 = -11\\
          &\hphantom{x_1 +} x_2 + 2x_3 + x_4 = 4
        \end{aligned}
      \right.
    \]
    
    Базисные переменные~---~которым соответствовали базисные столбцы в упрощённой матрице~---~можно выразить через свободные:
    \[
      \left\{
        \begin{aligned}
          &x_1 = 11x_3 + 5x_4 -11\\
          &x_2 = -2x_3 - x_4 + 4
        \end{aligned}
      \right.
    \]
    
    То есть при произвольных $x_3$ и $x_4$ рассчитанные по формулам выше $x_1$ и $x_2$ дадут в совокупности с $x_3$ и $x_4$ решение системы.
    Общий вид решения ($x_3 \hm\equiv t_1 \hm\in \RR$, $x_4 \hm\equiv t_2 \hm\in \RR$):
    \begin{equation*}
    \begin{split}
      \begin{pmatrix}
        x_1\\ x_2\\ x_3\\ x_4
      \end{pmatrix}
      = &\begin{pmatrix}
        11t_1 + 5t_2 - 11\\
        -2t_1 - t_2 + 4\\
        t_1\\
        t_2
      \end{pmatrix}\\
      = &\underbrace{\begin{pmatrix}
        11 \\ -2 \\ 1 \\ 0
      \end{pmatrix} t_1 + \begin{pmatrix}
        5 \\ -1 \\ 0 \\ 1
      \end{pmatrix} t_2}_{\substack{\tiny \mbox{Общее решение однородной системы}\\ \tiny \mbox{(решение при нулевом столбце свободных членов)}}} + \underbrace{\begin{pmatrix}
        -11 \\ 4 \\ 0 \\ 0
      \end{pmatrix}}_{\substack{\tiny \mbox{Частное решение неоднородной системы}\\ \tiny \mbox{(решение при нулевых свободных переменных)}}}\\
    = &\underbrace{\begin{pmatrix}
        11 & 5\\
        -2 & -1\\
        1 & 0\\
        0 & 1
      \end{pmatrix}}_{\substack{\tiny \mbox{Фундаментальная матрица}\\ \tiny \mbox{(её столбцы~---~базис в пространстве} \\ \tiny \mbox{решений однородной системы)}}} \begin{pmatrix} t_1 \\ t_2 \end{pmatrix} + \begin{pmatrix}
        -11 \\ 4 \\ 0 \\ 0
      \end{pmatrix}
    \end{split}
    \end{equation*}
  \end{solution}
  
  
  \subsection{\# 18.17(2)}
  
  Найти однородную систему, для которой фундаментальной является матрица $\Phi$ следующего вида:
  \[
    \Phi = \begin{pmatrix}
      1 & 0\\
      1 & -1\\
      0 & 1\\
      -4 & 1
    \end{pmatrix}
  \]
  
  \begin{solution}
    \hphantom{X}\par  % TODO: new line after "`Solution"'
    
    \emph{Способ 1}
    
    При данной фундаментальной матрице общее решение однородной системы выражается как линейная комбинация её столбцов:
    \begin{equation*}
    \begin{split}
      x = \Phi h = \begin{pmatrix}
          1 & 0\\
          1 & -1\\
          0 & 1\\
          -4 & 1
        \end{pmatrix} \begin{pmatrix}
          h_1 \\ h_2
        \end{pmatrix}
      = h_1 \cdot \begin{pmatrix} 1 \\ 1 \\ 0 \\ -4 \end{pmatrix} +
         h_2 \cdot \begin{pmatrix} 0 \\ -1 \\ 1 \\ 1 \end{pmatrix} \quad h_1, h_2 \in \RR
    \end{split}
    \end{equation*}
    
    Если переписать выражение для общего решения выше в виде системы:
    \[
      \left\{
        \begin{aligned}
          &x_1 = h_1\\
          &x_2 = h_1 - h_2\\
          &x_3 = h_2\\
          &x_4 = -4h_1 + h_2
        \end{aligned}
      \right. \Leftrightarrow \left\{
        \begin{aligned}
          &x_1 = h_1\\
          &x_2 = x_1 - x_3\\
          &x_3 = h_2\\
          &x_4 = -4x_1 + x_3
        \end{aligned}
      \right. \Leftrightarrow \left\{
        \begin{aligned}
          &x_2 = x_1 - x_3\\
          &x_4 = -4x_1 + x_3
        \end{aligned}
      \right. \Leftrightarrow \left\{
        \begin{aligned}
          &x_1 - x_2 - x_3 \hphantom{+ x_4} = 0\\
          &4x_1 \hphantom{+ x_2} - x_3 + x_4 = 0
        \end{aligned}
      \right.
    \]
    
    Очевидно, ответ не однозначен: к одной из строчек можно прибавить другую, и получится формально другая система.
    
    \bigskip
    
    \emph{Способ 2}
    
    В условии дана фундаментальная матрица.
    То есть её столбцы~---~решения однородной системы.
    Это значит, что в уравнение $Ax \hm= 0$ можно подставить вместо $x$ поочерёдно столбцы $\Phi$, и это будет давать верные числовые равенства.
    Пусть в матрице $A$ всего $m$ строк (и $4$ столбца).
    Тогда $Ax \hm= 0$ в виде системы можно записать так:
    \[
      \left\{
        \begin{aligned}
          &a_{11} x_1 + a_{12} x_2 + a_{13} x_3 + a_{14} x_4 = 0\\
          &a_{21} x_1 + a_{22} x_2 + a_{23} x_3 + a_{24} x_4 = 0\\
          &\ldots\\
          &a_{m1} x_1 + a_{m2} x_2 + a_{m3} x_3 + a_{m4} x_4 = 0
        \end{aligned}
      \right.
    \]
    
    Подставляем сюда вместо $x_1, \ldots, x_4$ компоненты первого столбца $\Phi$:
    \[
      \left\{
        \begin{aligned}
          &a_{11} + a_{12} \hphantom{+ a_{13}} - 4 a_{14} = 0\\
          &a_{21} + a_{22} \hphantom{+ a_{23}} - 4 a_{24} = 0\\
          &\ldots
        \end{aligned}
      \right.
    \]
    
    И компоненты второго столбца $\Phi$:
    \[
      \left\{
        \begin{aligned}
          &\hphantom{a_{11}} - a_{12} + a_{13} + a_{14} = 0\\
          &\hphantom{a_{21}} - a_{22} + a_{23} + a_{24} = 0\\
          &\ldots
        \end{aligned}
      \right.
    \]
    
    Отсюда надо найти коэффициенты $a_{ij}$, составляющие матрицу $A$.
    Чтобы это сделать, можно сгруппировать уравнения из двух систем по строчкам:
    \[
      \left\{
        \begin{aligned}
          &\left\{
            \begin{aligned}
              &a_{11} + a_{12} \hphantom{+ a_{13}} - 4 a_{14} = 0\\
              &\hphantom{a_{11}} - a_{12} + a_{13} + a_{14} = 0\\
            \end{aligned}
          \right.\\
          &\ldots
        \end{aligned}
      \right.
    \]
    
    В каждой такой паре уравнений можно принять первую и третью переменную за базисные (выразить через вторую и четвёртую).
    В итоге строки матрицы $A$ должны выглядеть так:
    \[
      A = \begin{pmatrix}
        &(-a_{12} + 4 a_{14}) + a_{12} + (a_{12} - a_{14}) + a_{14}\\
        &(-a_{22} + 4 a_{24}) + a_{22} + (a_{22} - a_{24}) + a_{24}\\
        &\ldots
      \end{pmatrix}
    \]
    
    Четыре столбца линейно зависимы: первый и третий выражаются через второй и четвёртый.
    Поэтому максимальное число линейно независимых строк в матрице $A$ равно двум.
    Меньше двух строк (одной) быть не может, так как число столбцов в фундаментальной матрице $2 \hm= 4 \hm- \Rg A$.
    Поэтому остаётся составить две линейно независимые строки коэффициентов, удовлетворяющие соотношениям выше.
    Например,
    \[
      A = \begin{pmatrix}
        1 & -1 & -1 & 0\\
        4 & 0 & -1 & 1
      \end{pmatrix}
    \]
    
    И тогда система уравнений:
    \[
      \left\{
        \begin{aligned}
          &x_1 - x_2 - x_3 \hphantom{+ x_4} = 0\\
          &4x_1 \hphantom{+ x_2} - x_3 + x_4 = 0
        \end{aligned}
      \right.
    \]
  \end{solution}
\end{document}
