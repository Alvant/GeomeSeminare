\documentclass[a4paper,12pt]{article}

\usepackage{mystyle}
\usepackage{gensymb}


\usepackage{scalerel}
\usepackage{stackengine}

\graphicspath{ {images/} }


\definecolor{violet}{RGB}{148, 0, 211}


% https://tex.stackexchange.com/a/101138/135045

\newcommand\widesim[1]{\ThisStyle{%
  \setbox0=\hbox{$\SavedStyle#1$}%
  \stackengine{-.1\LMpt}{$\SavedStyle#1$}{%
    \stretchto{\scaleto{\SavedStyle\mkern.2mu\sim}{.5150\wd0}}{.6\ht0}%
  }{O}{c}{F}{T}{S}%
}}

\newcommand{\BigMiddleThree}{\;\left|\vphantom{\begin{pmatrix} 0\\0\\0 \end{pmatrix}}\right.\;}
\newcommand{\BigMiddleFour}{\;\left|\vphantom{\begin{pmatrix} 0\\0\\0\\0 \end{pmatrix}}\right.\;}


\author{Алексеев Василий}


\title{Семинар 3}
\date{18 февраля + 19 февраля 2021}


\begin{document}
  \maketitle
  
  \tableofcontents

  \thispagestyle{empty}
  
  \newpage
  
  \pagenumbering{arabic}


  \section{Линейные пространства: подпространства, базис}
  
  \section{Задачи}
  
  
  \subsection{\# 20.3(4)}
  
  Выяснить, является ли линейным подпространством следующее множество векторов в $n$-мерном пространстве:
  множество векторов $\mathcal V$, сумма координат которых равна $1$.
  
  \begin{solution}
    Пусть $\bds a, \bds b \in \mathcal V$.
    При этом $\bds a \hm= (a_1, \ldots, a_n)$, $\bds b \hm= (b_1, \ldots, b_n)$\footnote{Как часто уже было и ещё будет, равенство между вектором и столбцом из компонент в данном случае~---~это не совсем ``равенство''. Это значит, что множество векторов изоморфно множеству столбцов, состоящих из компонент векторов в фиксированном базисе.}.
    Тогда сумма векторов
    \[
      \bds a + \bds b = (a_1 + b_1, \ldots, a_n + b_n)
    \]
    
    И сумма координат у суммы:
    \[
      \sum_i (a_i + b_i) = \sum_i a_i + \sum_i b_i = 1 + 1 = 2 \not= 1
    \]
    
    $\mathcal V$ не замкнуто относительно суммы векторов (при умножении вектора на число тоже можно получить ``не то''), поэтому не является подпространством.
  \end{solution}
  
  
  \subsection{\# 20.18}
  
  Доказать, что четыре матрицы
  \begin{equation}\label{p-20-18-matrices}
    \left\{
      \begin{pmatrix}
        1 & -1\\
        1 & 1
      \end{pmatrix},
      \begin{pmatrix}
        2 & 5\\
        1 & 3
      \end{pmatrix},
      \begin{pmatrix}
        1 & 1\\
        0 & 1
      \end{pmatrix},
      \begin{pmatrix}
        3 & 4\\
        5 & 7
      \end{pmatrix}
    \right\}
  \end{equation}
  образуют базис в пространстве квадратных матриц порядка $2$.
  
  Далее найти компоненты матрицы
  \[
    \begin{pmatrix}
      5 & 14\\
      6 & 13
    \end{pmatrix}
  \]
  в этом базисе.
  
  \begin{solution}
    Покажем, что четыре матрицы (\ref{p-20-18-matrices}) в самом деле можно взять в качестве базиса.
    Для этого можно просто проверить, как по системе раскладывается нулевая матрица (если единственной решение~---~тривиальное, то система линейно независима).
    Итак, составляем линейную комбинацию матриц (\ref{p-20-18-matrices}) и приравниваем её нулевой:
    \[
      \alpha \begin{pmatrix}
        1 & -1\\
        1 & 1
      \end{pmatrix} + \beta \begin{pmatrix}
        2 & 5\\
        1 & 3
      \end{pmatrix} + \gamma \begin{pmatrix}
        1 & 1\\
        0 & 1
      \end{pmatrix} + \zeta \begin{pmatrix}
        3 & 4\\
        5 & 7
      \end{pmatrix} = 0
    \]
  \end{solution}
  
  Составляем систему:
  \[
    \left\{
      \begin{aligned}
        &\alpha + 2\beta + \gamma + 3\zeta = 0\\
        &\alpha + \beta \hphantom{+ \gamma} + 5\zeta = 0\\
        &-\alpha + 5\beta + \gamma + 4\zeta = 0\\
        &\alpha + 3\beta + \gamma + 7\zeta = 0
      \end{aligned}
    \right.
  \]
  
  И решаем её методом Гаусса (на каждом шаге цветом выделен элемент, который используем для ``устранения'' других ненулевых в том же столбце\footnote{И каждый раз такой элемент выбирается в новой строке.}):
  
  \begin{equation}\label{p-20-18-independence}
  \begin{split}
    \begin{pmatrix}
      \textcolor{violet}{\bds 1} & 2 & 1 & 3\\
      1 & 1 & 0 & 5\\
      -1 & 5 & 1 & 4\\
      1 & 3 & 1 & 7
    \end{pmatrix}
    &\sim \begin{pmatrix}
      1 & 2 & 1 & 3\\
      0 & \textcolor{violet}{\bds{-1}} & -1 & 2\\
      0 & 7 & 2 & 7\\
      0 & 1 & 0 & 4
    \end{pmatrix}
    \sim \begin{pmatrix}
      1 & 0 & -1 & 7\\
      0 & -1 & -1 & 2\\
      0 & 0 & -5 & 21\\
      0 & 0 & \textcolor{violet}{\bds{-1}} & 6
    \end{pmatrix}\\
    &\sim \begin{pmatrix}
      1 & 0 & 0 & 1\\
      0 & -1 & 0 & -4\\
      0 & 0 & 0 & -9\\
      0 & 0 & -1 & 6
    \end{pmatrix}
    \sim \begin{pmatrix}
      1 & 0 & 0 & 1\\
      0 & -1 & 0 & -4\\
      0 & 0 & 0 & \textcolor{violet}{\bds 1}\\
      0 & 0 & -1 & 6
    \end{pmatrix}
    \sim \begin{pmatrix}
      1 & 0 & 0 & 0\\
      0 & -1 & 0 & 0\\
      0 & 0 & 0 & 1\\
      0 & 0 & -1 & 0
    \end{pmatrix}
  \end{split}
  \end{equation}
  
  Таким образом, векторы-матрицы линейно независимы\footnote{
    Можно бы было доказывать линейную независимость не ``через матрицы'', а ``через вектора'' (если так приятнее).
    То есть каждую матрицу из (\ref{p-20-18-matrices}) можно бы было ``развернуть'' в столбец.
    Например, первой матрице $\left(\begin{smallmatrix} 1 & -1 \\ 1 & 1\end{smallmatrix}\right)$ можно бы было сопоставить столбец $\left(\begin{smallmatrix} 1 & -1 & 1 & 1\end{smallmatrix}\right)^T$ (и оставшимся матрицам~---~по такому же правилу).
    После этого, возможно, было бы немного ``приятнее'' составлять линейную комбинацию из векторов-столбцов и приравнивать её нулевому столбцу.
    В итоге же всё равно можно бы было прийти к системе линейных уравнений.
    Линейная же зависимость матриц и ``развёрнутых столбцов'' равносильны, потому что множества матриц второго порядка и столбцов размера четыре (которые сопоставляются матрицам по конкретному правилу) изоморфны.
  }.
  Их четыре (в четырёхмерном пространстве), поэтому это~---~полная система.
  Поэтому их можно взять в качестве базиса.
  
  \bigskip
  
  Чтобы теперь разложить матрицу $\left(\begin{smallmatrix} 5 & 14\\ 6 & 13 \end{smallmatrix}\right)$ по базису, надо представить её в виде их линейной комбинации (решение точно будет существовать, притом единственное, так как система матриц (\ref{p-20-18-matrices})~---~базис):
  \[
    \alpha \begin{pmatrix}
      1 & -1\\
      1 & 1
    \end{pmatrix} + \beta \begin{pmatrix}
      2 & 5\\
      1 & 3
    \end{pmatrix} + \gamma \begin{pmatrix}
      1 & 1\\
      0 & 1
    \end{pmatrix} + \zeta \begin{pmatrix}
      3 & 4\\
      5 & 7
    \end{pmatrix} = \begin{pmatrix}
      5 & 14\\
      6 & 13
    \end{pmatrix}
  \]
  
  Расширенная матрица системы:
  \[
    \left(
      \begin{matrix}
        1 & 2 & 1 & 3\\
        1 & 1 & 0 & 5\\
        -1 & 5 & 1 & 4\\
        1 & 3 & 1 & 7
      \end{matrix}
      \BigMiddleFour
      \begin{matrix}
        5 \\ 6 \\ 14 \\ 13
      \end{matrix}
    \right)
  \]
  
  Далее с расширенной матрице можно провести те же элементарные преобразования строк, что и в случае (\ref{p-20-18-independence}).
  При этом можно работать только с последним столбцом (так как для матрицы уже ``всё сделано''):
  \begin{equation*}
  \begin{split}
    \begin{pmatrix} 5 \\ 6 \\ 14 \\ 13 \end{pmatrix}
    \sim \begin{pmatrix} 5 \\ 1 \\ 19 \\ 8 \end{pmatrix}
    \sim \begin{pmatrix} 7 \\ 1 \\ 26 \\ 9 \end{pmatrix}
    \sim \begin{pmatrix} -2 \\ -8 \\ -19 \\ 9 \end{pmatrix}
    \sim \begin{pmatrix} -2 \\ -8 \\ 19/9 \\ 9 \end{pmatrix}
    \sim \begin{pmatrix} -37/9 \\ 4/9 \\ 19/9 \\ -33/9 \end{pmatrix}
  \end{split}
  \end{equation*}
  
  Поэтому коэффициенты в разложении
  \[
    (\alpha, \beta,  \gamma, \zeta) = \left(-\frac{37}{9}, -\frac{4}{9}, \frac{33}{9}, \frac{19}{9}\right)
  \]
  
  
  \subsection{\# 20.22(3)}
  
  Найти размерность и базис подпространства\footnote{Система $Ax \hm= 0$ в самом деле определяет подпространство: можно проверить замкнутость относительно сложения векторов и умножения вектора на число.}
  \[
    \left\{
      \begin{pmatrix}
        x_1 \\ x_2 \\ x_3
      \end{pmatrix}
      \BigMiddleThree
      \begin{pmatrix}
        -3 & 1 & -2\\
        6 & -2 & 4\\
        -15 & 5 & -10
      \end{pmatrix}
      \begin{pmatrix}
        x_1 \\ x_2 \\ x_3
      \end{pmatrix}
      = 0
    \right\}
  \]
  
  \begin{solution}
    Приводим матрицу к упрощённому виду, выражаем базисные переменные через свободные:
    \[
      \begin{pmatrix}
        -3 & 1 & -2\\
        6 & -2 & 4\\
        -15 & 5 & -10
      \end{pmatrix}
      \sim \begin{pmatrix}
        -3 & 1 & -2\\
        0 & 0 & 0\\
        0 & 0 & 0
      \end{pmatrix}
      \leftrightarrow -3x_1 + x_2 - 2x_3 = 0
      \Leftrightarrow x_2 = 3x_1 + 2x_3
    \]
    
    Таким образом, решение системы:
    \[
      \begin{pmatrix}
        \alpha \\ 3\alpha + 2\gamma \\ \gamma
      \end{pmatrix}
      = \alpha \begin{pmatrix}
        1 \\ 3 \\ 0
      \end{pmatrix} + \gamma \begin{pmatrix}
        0 \\ 2 \\ 1
      \end{pmatrix}
    \]
    
    Вектора базиса найдены, их два, поэтому размерность подпространства равна двум.
  \end{solution}
\end{document}
