\documentclass[a4paper,12pt]{article}

\usepackage{mystyle}
\usepackage{gensymb}


\usepackage{scalerel}
\usepackage{stackengine}

\graphicspath{ {images/} }


% https://tex.stackexchange.com/questions/5461/is-it-possible-to-change-the-size-of-an-arrowhead-in-tikz-pgf
\usetikzlibrary{arrows.meta}


\DeclareMathOperator{\Image}{Im}
\definecolor{violet}{RGB}{148, 0, 211}
\definecolor{green}{RGB}{0, 153, 0}
\definecolor{orange}{RGB}{255, 153, 0}
\definecolor{blue}{RGB}{31, 117, 254}


% https://tex.stackexchange.com/a/101138/135045

\newcommand\widesim[1]{\ThisStyle{%
  \setbox0=\hbox{$\SavedStyle#1$}%
  \stackengine{-.1\LMpt}{$\SavedStyle#1$}{%
    \stretchto{\scaleto{\SavedStyle\mkern.2mu\sim}{.5150\wd0}}{.6\ht0}%
  }{O}{c}{F}{T}{S}%
}}

\newcommand{\BigMiddleThree}{\;\left|\vphantom{\begin{pmatrix} 0\\0\\0 \end{pmatrix}}\right.\;}
\newcommand{\BigMiddleFour}{\;\left|\vphantom{\begin{pmatrix} 0\\0\\0\\0 \end{pmatrix}}\right.\;}


% https://tex.stackexchange.com/questions/63531/how-to-write-quotation-marks-in-math-environment
\DeclareMathSymbol{\mlq}{\mathord}{operators}{``}
\DeclareMathSymbol{\mrq}{\mathord}{operators}{`'}


\DeclareMathOperator{\Imag}{Im}


% https://tex.stackexchange.com/questions/544453/undefined-control-sequence-after-paragraph
\renewcommand{\paragraph}[1]{\noindent\textbf{#1}\quad}




\author{Алексеев Василий}


\title{Семинар 6}
\date{10 марта + 14 марта 2022}


\begin{document}
  \maketitle
  
  \tableofcontents

  \thispagestyle{empty}
  
  \newpage
  
  \pagenumbering{arabic}


  \section{Линейные отображения $2$}
  
  \subsection{Сюжет 1: Матрица линейного отображения}
  \label{sec:plot1}
  
  Пусть есть линейное отображение $\phi \hm\colon X \hm\to Y$, где $X$ и $Y$~---~линейные пространства (см. Рис.~\ref{fig:map-maps-map-to-map}).
  Выберем базисы в $X$ и $Y$: $e \hm= (\bds e_1, \ldots, \bds e_n) \hm\subset X$, $f \hm= (\bds f_1, \ldots, \bds f_m) \hm\subset Y$, при этом будем считать $n \hm> 0$ и $m \hm> 0$.
  Рассмотрим действие отображения $\phi$ на вектор $\bds x \hm\in X$ с компонетнами $(x_1, \ldots, x_n)$ в базисе $e$:
  \begin{equation*}
  \begin{split}
    \phi(\bds x) = \phi(x_1 \bds e_1 + \ldots + x_n \bds e_n)
    = x_1 \phi(\bds e_1) + \ldots + x_n \phi(\bds e_n)
    = \underbrace{\bigl(\phi(\bds e_1), \ldots, \phi(\bds e_n)\bigr)}_{\mbox{строка векторов}} \underbrace{\begin{pmatrix}
      x_1 \\ \vdots \\ x_n
    \end{pmatrix}}_{\mbox{столбец координат}}
  \end{split}
  \end{equation*}
  
  Каждый из векторов $\phi(\bds e_i) \hm\in Y$ можно разложить по базису $f$.
  Например
  \[
    \phi(\bds e_1) \hm= (\bds f_1, \ldots, \bds f_m) \phi(e_1)
  \]
  где как $\phi(e_1) \hm\in \RR^m$ обозначен вектор-столбец вектора $\phi(\bds e_1) \hm\in Y$\footnote{В обозначениях Рис.~\ref{fig:map-maps-map-to-map} $\phi(e_1)$ это $\widetilde\phi\bigl( h_X(\bds e_1) \bigr)$.}.
  Таким образом,
  \begin{equation*}
    \phi(\bds x) = \ldots
    = (\bds f_1, \ldots, \bds f_m) \underbrace{\bigl(\phi(e_1), \ldots, \phi(e_n)\bigr)}_{\in \RR^{m \times n}} \begin{pmatrix}
      x_1 \\ \vdots \\ x_n
    \end{pmatrix}
  \end{equation*}
  
  С другой стороны, так как вектор $\phi(\bds x) \hm\in Y$, то он раскладывается по базису $f$ с некоторыми коэффициентами $y_1, \ldots, y_m$:
  \[
    \phi(\bds x) = (\bds f_1, \ldots, \bds f_m) \begin{pmatrix}
      y_1 \\ \vdots \\ y_m
    \end{pmatrix}
  \]
  
  Получили, что
  \[
    (\bds f_1, \ldots, \bds f_m) \bigl(\phi(e_1), \ldots, \phi(e_n)\bigr) \begin{pmatrix}
      x_1 \\ \vdots \\ x_n
    \end{pmatrix}
    = (\bds f_1, \ldots, \bds f_m) \begin{pmatrix}
      y_1 \\ \vdots \\ y_m
    \end{pmatrix}
  \]
  
  Так как $(\bds f_1, \ldots, \bds f_m)$ базис, то:
  \[
    \bigl(\phi(e_1), \ldots, \phi(e_n)\bigr) \begin{pmatrix}
      x_1 \\ \vdots \\ x_n
    \end{pmatrix} = \begin{pmatrix}
      y_1 \\ \vdots \\ y_m
    \end{pmatrix}
  \]
  
  Если ввести обозначения $\bigl(\phi(e_1), \ldots, \phi(e_n)\bigr) \hm\equiv A$~---~\emph{матрица линейного отображения},
  $\left(\begin{smallmatrix}x_1 \\ \vdots \\ x_n\end{smallmatrix}\right) \hm\equiv \xi$~---~вектор-столбец прообраза,
  $\left(\begin{smallmatrix}y_1 \\ \vdots \\ y_m\end{smallmatrix}\right) \hm\equiv \eta$~---~вектор-столбец образа,
  то выражение выше будет выглядеть как
  \[
    \boxed{\eta = A \xi}
  \]
  
  \begin{figure}[h]
    \centering
    
    \begin{tikzpicture}
      \node[anchor=south] at (0,0) {$\bds x$};
      \coordinate (A) at (0,0);
      
      \node[anchor=south] at (-0.6,0) {$X \ni$};  % TODO: align text in node
      
      \node[anchor=south] at (0.8,0) {$\overset{\phi}{\mapsto}$};
      
      \node[anchor=south] at (1.6,-0.05) {$\bds y$};
      \coordinate (B) at (1.6,-0.05);
      
      \node[anchor=south] at (2.2,0) {$\in Y$};
      
      
      \node[anchor=north] at (0,-1.55) {$\xi$};
      \coordinate (C) at (0,-1.55);
      
      \node[anchor=north] at (-0.7,-1.5) {$\RR^n \ni$};
      
      \node[anchor=north] at (0.8,-1.6) {$\underset{\widetilde \phi}{\mapsto}$};
      
      \node[anchor=north] at (1.6,-1.6) {$\eta$};
      \coordinate (D) at (1.6,-1.6);
      
      \node[anchor=north] at (2.3,-1.5) {$\in \RR^m$};
      
      
      \draw[|-Stealth,thick] (A) -- (C) node [midway, left] {\footnotesize $h_X$};
      \draw[|-Stealth,thick] (B) -- (D) node [midway, right] {\footnotesize $h_Y$};
    \end{tikzpicture}
    
    \caption{Линейное отображение $\phi \hm\colon X \hm\to Y$, действующее из линейного пространства $X$ размерности $n$ в линейное пространство $Y$ размерности $m$.
      Выбор базиса $e \hm= (e_1, \ldots, e_n)$ в пространстве $X$ порождает отображение $h_X$, переводящее вектор $x \hm\in X$ в его координатный столбец $\xi \hm\in \RR^n$.
      Аналогично, выбор базиса $f \hm= (f_1, \ldots, f_m)$ в пространстве $Y$ порождает отображение $h_Y$, переводящее вектор $y \hm\in Y$ в его координатный столбец $\eta \hm\in \RR^m$.
      (Можно заметить, что $h_X$ и $h_Y$~---~биекции, причём такие, которые сохраняют линейные операции: суммы и умножения на число.)
      Таким образом, выбор пары базисов $e$ и $f$ в пространствах $X$ и $Y$ порождает отображение $\widetilde \phi$, переводящее вектор-столбец $\xi$ в вектор-столбец $\eta$ ($\widetilde \phi \hm= h_Y \phi h_X^{-1}$).
      При этом $\boxed{\eta \hm= A \xi}$, где $A \hm\in \RR^{m \times n}$~---~\emph{матрица линейного отображения}~$\phi$.
      (Матрица определяется выбором базисов в пространствах $X$ и $Y$.)
    }
    \label{fig:map-maps-map-to-map}
  \end{figure}
  
  
  \subsection{Сюжет 2: Ранг матрицы линейного отображения}
  
  Пусть в $X$ и $Y$ выбраны базисы, и матрица $A$~---~матрица отображения $\phi$.
  Множество столбцов
  \[
    I = \{\eta \in \RR^m \mid \eta = A \xi,\ \xi \in \RR^n\}
  \]
  будет линейным подпространством $\RR^m$ и представляет совокупность вектор-столбцов элементов из $\Imag \phi$.
  Размерности $I$ и $\Imag \phi$ совпадают, потому что сумма векторов $\bds y_1$ и $\bds y_2$ из $\Imag \phi$ есть вектор, координатный столбец которого есть сумма координатных столбцов векторов $\bds y_1$ и $\bds y_2$.
  Аналогично с умножением на число.
  То есть линейные операции ``проходят одинаково'': между векторами в $\Imag \phi$ и их координатными столбцами в $I$.
  
  Запись $\eta \hm= A \xi$ означает, что $\eta$ есть линейная комбинация столбцов $A$ с коэффициентами, записанными в столбец $\xi$.
  Пусть столбцовый ранг матрицы $A$ равен $r \hm\leq n$.
  Тогда все оставшиеся $n \hm- r$ столбцов $A$ можно выразить через базисные, и произвольный $\eta \hm\in I$ окажется представленным как линейная комбинация $r$ линейно независимых вектор-столбцов из $I$.
  Таким образом, $\dim I \hm= r \hm= \Rg A$.
  В то же время $\dim I \hm= \dim \Imag\phi \hm= \Rg \phi$.
  Получаем, что
  \[
    \boxed{\Rg A = \Rg \phi}
  \]
  то есть ранг матрицы линейного отображения равен рангу этого отображения.
  Матрица $A$ зависит от выбора базисов в пространствах $X$ и $Y$, но ранг её не меняется и равен рангу отображения.

  
  \subsection{Сюжет 3: Пространство отображений}
  \label{sec:plot2}
  
  Рассмотрим множество всех линейных отображений из $X$ в $Y$:
  \[
    \mathcal F = \{\phi \mid \phi\colon X \to Y,\ \phi\mbox{~---~линейное}\}
  \]
  где линейность $\phi$ обозначает выполнение следующих двух свойств:
  \begin{itemize}
    \item $\phi(\bds x_1 + \bds x_2) = \phi(\bds x_1) + \phi(\bds x_2),\quad \bds x_1, \bds x_2 \in X$
    \item $\phi(\alpha \bds x_1) = \alpha \phi(\bds x_1),\quad \alpha \in \RR,\ \bds x_1 \in X$
  \end{itemize}
  
  Введём на $\mathcal F$ операции сложения отображений и умножения отображения на число.
  За сумму $\phi_1 \hm+ \phi_2$ двух линейных отображений $\phi_1$ и $\phi_2$ будем считать отображение из $X$ в $Y$, которое действует на произвольный $\bds x \hm\in X$ как:
  \[
    (\phi_1 + \phi_2) (\bds x) \equiv \phi_1(\bds x) + \phi_2(\bds x)
  \]
  
  За отображение $\alpha\phi$, где $\alpha \hm\in \RR$ и $\phi \hm\in \mathcal F$, будет считать отображение из $X$ в $Y$, действующее на произвольный $\bds x \hm\in X$ по правилу:
  \[
    (\alpha \phi) (\bds x) \equiv \alpha \phi(\bds x)
  \]
  
  Проверим, что сумма линейных отображений~---~это тоже линейное отображение:
  \begin{equation*}
  \begin{split}
    (\phi_1 + \phi_2) (\bds x_1 + \bds x_2)
    &= \phi_1 (\bds x_1 + \bds x_2) + \phi_2 (\bds x_1 + \bds x_2)
    = \phi_1 (\bds x_1) + \phi_1 (\bds x_2) + \phi_2 (\bds x_1) + \phi_2 (\bds x_2) \\
    &= \bigl(\phi_1(\bds x_1) + \phi_2(\bds x_1)\bigr) + \bigl(\phi_1(\bds x_2) + \phi_2(\bds x_2)\bigr)
    = (\phi_1 + \phi_2) (\bds x_1) + (\phi_1 + \phi_2) (\bds x_2)
  \end{split}
  \end{equation*}
  
  Аналогично можно показать, что $(\phi_1 \hm+ \phi_2) (\alpha \bds x) \hm= \alpha (\phi_1 \hm+ \phi_2) (\bds x)$.
  Таким образом, линейные операции над линейными отображениями из $\mathcal F$ дают также линейные отображения из $\mathcal F$,
  то есть $``{+}''\colon \mathcal F \hm\times \mathcal F \hm\to \mathcal F$ и $``{\cdot}'' \colon \RR \hm\times \mathcal F \hm\to \mathcal F$.
  
  Множество $\mathcal F$ с введёнными операциями $``{+}''$ и $``{\cdot}''$ образует линейное пространство.
  Чтобы в этом убедиться, можно проверить выполнение ``$8$ свойств'', связынных с операциями сложения и умножения на число.
  Например, коммутативность: $(\phi_1 \hm+ \phi_2)(\bds x) \hm= \phi_1(\bds x) \hm+ \phi_2(\bds x) \hm= \phi_2(\bds x) \hm+ \phi_1(\bds x) \hm= (\phi_2 \hm+ \phi_1)(\bds x)$ для произвольного $\bds x \hm\in X$.
  Нулевым же, очевидно, будет отображение $\phi_0\colon \bds x \hm\mapsto \bds 0 \hm\in Y$.
  
  
  \subsection{Сюжет 4: Изоморфизм}
  \label{sec:iso}
  
  В выбранной паре базисов пространств $X$ и $Y$ устанавливается взаимно однозначное соответствие $h_{\mathcal F}$ между линейными отображениями $\mathcal F$ и матрицами $\RR^{m \times n}$ (см. Раздел~\ref{sec:plot1}).
  Множество матриц $\RR^{m \times n}$~---~очевидно, линейное пространство (размерности $mn$).
  Множество линейных отображений $\mathcal F$~---~также линейное пространство (см. Раздел~\ref{sec:plot2}).
  При этом сумме отображений $\phi_1$ и $\phi_2$ из $\mathcal F$ с матрицами соответственно $A_1$ и $A_2$ соответствует отображение, матрица которого является суммой $A_1 \hm+ A_2$.
  Умножению отображения $\phi$ с матрицей $A$ на число $\alpha$ соответствует отображение с матрицей $\alpha A$.
  Таким образом, отображение $h_{\mathcal F}\colon \mathcal F \hm\to \RR^{m \times n}$ линейно, \emph{сохраняет линейные операции}.
  Взаимно однозначное линейное отображение называется \emph{изоморфизмом}\footnote{В общем случае изоморфизм~---~биекция, \emph{сохраняющая структуру}. В случае отображений между линейными пространствами изоморфизм должен сохранять линейные зависимости между векторами: сумма прообразов~---~сумма образов, аналогично с умножением на число.}.
  Про пространства $\mathcal F$ и $\RR^{m\times n}$ в таком случае говорят, что они \emph{изоморфны}.
  Из изоморфности $\mathcal F$ и $\RR^{m\times n}$, в частности, следует, что размерность $\mathcal F$ равна также $mn$.
  
  
  
  \subsection{Последний сюжет: Линейные функции (тема семинара)}
  
  Пусть $Y \hm\equiv \RR$.
  То есть будем теперь рассматривать линейные отображения, действующие из $X$ в $\RR$:
  \[
    \mathcal F_1 = \{\phi \mid \phi\colon X \to \RR,\ \phi\mbox{~---~линейное}\}
  \]
  Такие отображения ещё называют \emph{линейными функциями}.
  
  С выбранным базисом в $X$\footnote{В $Y \hm= \RR$ тоже можно бы было ``выбрать'' базис, но обычно базисом в $\RR$ по умолчанию считают единицу.} каждой линейной функции из $\mathcal F_1$ ставится в соответствие матрица отображения $A$ размера $1 \hm\times n$, то есть матрица отображения в случае линейной функции~---~это одна строка.
  Эта строка ещё называется \emph{координатной строкой} линейной функции.
  
  Пространство линейных функций $\mathcal F_1$ изоморфно пространству строк $\RR^n$ длины $n$ (см.~Раздел~\ref{sec:iso}).
  В пространстве строк можно очевидным образом выбрать базис: $(1, 0, \ldots, 0, 0)$; ...; $(0, 0, \ldots, 0, 1)$.
  Пусть базисной строчке из $\RR^n$ с единицей на $i$-ой позиции соотвествует функция $\phi_i \hm\in \mathcal F_1$:
  \[
    \RR^n \ni \left\{
      \begin{aligned}
        &(1, 0, \ldots, 0, 0)            & &\leftrightarrow & &\phi_{1\hphantom{-1}}\\
        &(0, 1, \ldots, 0, 0)            & &\leftrightarrow & &\phi_{2\hphantom{-1}}\\
        &\hphantom{(0, 1, \ldots, 0, 0)} & &\ldots  & &\hphantom{\phi_{2\hphantom{-1}}}\\
        &(0, 0, \ldots, 1, 0)            & &\leftrightarrow & &\phi_{n-1}\\
        &(0, 0, \ldots, 0, 1)            & &\leftrightarrow & &\phi_{n\hphantom{-1}}
      \end{aligned}
    \right\} \in \mathcal F_1
  \]
  В совокупности функции $(\phi_1, \phi_2, \ldots, \phi_n)$ дадут базис в $\mathcal F_1$.
  Этот базис называется базисом, \emph{взаимным} к базису $e$ пространства $X$.
  Само пространство линейных функций, действующих на пространстве $X$, называется пространством, \emph{сопряжённым} к $X$, и может обозначаться как $X^{*}$.
  Таким образом, $\dim X^{*} \hm= \dim X$, а смысл \emph{координатной строки} функции $\phi$ также в том, что она~---~это коэффициенты разложения $\phi$ по взаимному базису.
  
  
  
  \section{Задачи}
  
  
  \subsection{\# 23.9(2)}
  
  Найти матрицу следующего преобразования $\phi \colon \mathcal E \hm\to \mathcal E$ векторов трёхмерного геометрического пространства:
  $\phi$~---ортогональное проектирование на $\mathcal L_1 \hm= \{\bds v \hm\in \mathcal E \hm\mid \bds v \hm= \alpha \bds a,\ \alpha \hm\in \RR\}$, где $\bds a \hm= (1, 1, 1)^T$.
  
  \begin{solution}
    Формула, задающая преобразование:
    \[
      \phi(\bds v) = \frac{\bds a}{|\bds a|} \frac{(\bds a, \bds v)}{|\bds a|}
      = \frac{\bds a}{|\bds a|^2} (\bds a, \bds v) \in \mathcal E
    \]
    
    Преобразование задаёт связь между векторами, матрица преобразования связывает координатные столбцы.
    Распишем координатный столбец образа $\phi(v) \hm\in \RR^3$, где $v \hm= (x_1, x_2, x_3) \hm\in \RR^3$~---~координатный столбец прообраза:
    \[
      \phi(v) = \frac{(1, 1, 1)^T}{3} \cdot (x_1 + x_2 + x_3)
      = \underbrace{\frac{1}{3}\begin{pmatrix}
        1 & 1 & 1\\
        1 & 1 & 1\\
        1 & 1 & 1
      \end{pmatrix}}_{A} \begin{pmatrix}
        x_1 \\ x_2 \\ x_3
      \end{pmatrix}
    \]
    
    Можно бы было искать по отдельности столбцы $A$:
    \[
      A = \bigl(\phi(e_1), \phi(e_2), \phi(e_3)\bigr) = \frac{1}{3}\begin{pmatrix}
        1 & 1 & 1\\
        1 & 1 & 1\\
        1 & 1 & 1
      \end{pmatrix}
    \]
    
    Если же бы $\phi$ рассматривалось не как преобразование, а как отображение $\widetilde \phi\colon \mathcal E \hm\to \mathcal L_1$ (и пусть при этом за базис в $\mathcal L_1$ естественным образом выбран вектор $\bds a$), то матрица $\widetilde A$ была бы такой (индексом $a$ обозначен базис, в котором составлен координатный столбец)
    \[
      \widetilde A \hm= \bigl(\phi(e_1)_a, \phi(e_2)_a, \phi(e_3)_a\bigr) \hm= (1/3, 1/3, 1/3)
    \]
  \end{solution}
  
  
  \subsection{\# 23.15(1)}
  
  Пусть линейное пространство $\mathcal L$ представимо как прямая сумма двух ненулевых подпространств: $\mathcal L \hm= \mathcal L_1 \oplus \mathcal L_2$.
  
  Показать, что преобразование $\phi\colon X \hm\to Y$, где $X \hm= Y \hm= \mathcal L$, проектирования на $\mathcal L_1$ параллельно $\mathcal L_2$ линейно.
  Найти ядро и множество значений $\phi$.
  Найти матрицу преобразования $\phi$ в базисе $\mathcal L$, составленном из базисов подпространств $\mathcal L_1$ и $\mathcal L_2$.
  
  \begin{solution}
    \textbf{Линейность.} Раз $\mathcal L$ выражено прямой суммой $\mathcal L_1$ и $\mathcal L_2$, то любой вектор $\bds x$ из $\mathcal L$ единственным образом раскладывается в сумму двух, один из которых в $\mathcal L_1$, а другой в $\mathcal L_2$:
    \[
      \mathcal L = \mathcal L_1 \oplus \mathcal L_2
      \Leftrightarrow \underbrace{\bds x_{\vphantom{1}}}_{\in \mathcal L} = \underbrace{\bds x_1}_{\in \mathcal L_1} + \underbrace{\bds x_2}_{\in \mathcal L_2} 
    \]  % TODO: first brace is bold for some reason...
    
    В таком представлении
    \[
      \phi(\bds x) = \bds x_1
    \]
    
    И можно проверить линейность преобразования:
    \[
      \phi(\bds x + \bds y) = \phi(\bds x_1 + \bds x_2 + \bds y_1 + \bds y_2)
      = \phi\bigl((\bds x_1 + \bds y_1) + (\bds x_2 + \bds y_2)\bigr)
      = \bds x_1 + \bds y_1 = \phi(\bds x) + \phi(\bds y)
    \]
    
    Аналогично $\phi(\alpha \bds x) \hm= \alpha \phi(\bds x)$.
    
    \medskip
    
    \textbf{Ядро} преобразования определяется как
    \[
      \phi(\bds x) = 0 \Leftrightarrow \bds x_1 = 0 \Leftrightarrow \bds x = \bds x_2 \in \mathcal L_2
    \]
    то есть $\Ker\phi \hm= \mathcal L_2$.
    
    \medskip
    
    \textbf{Множество значений} есть подмножество $\mathcal L$ векторов $\bds y$, которые могут быть получены с помощью преобразования $\phi$.
    То есть рассматриваем $\bds y \hm\in \mathcal L$ и проверяем, при каких условиях его можно получить с помощью $\phi$.
    Очевидно, если существует $\bds x$, являющийся прообразом $\bds y$, то
    \[
      \phi(\bds x) = \bds y \Rightarrow \bds y \in \mathcal L_1
    \]
    То есть $\Imag\phi \hm{\subseteq} \mathcal L_1$.
    Но верно и в другую сторону:
    \[
      \bds y \in \mathcal L_1 \Rightarrow \exists \bds x = \bds y \in \mathcal L: \phi(\bds x) = \bds y
    \]
    Поэтому $\mathcal L_1 \hm{\subseteq} \Imag\phi$, и в итоге $\Imag\phi \hm= \mathcal L_1$.
    
    Можно заметить, что выполняется тождество\footnote{Верно и в общем случае для линейного отображения $\phi \colon X \hm\to Y$. Доказательство можно свести к рассмотрению системы линейных уравнений $\eta_m \hm= A_{m\times n} \xi_n$. В фундаментальной матрице соответствующей однородной системы будет $n \hm- r$ столбцов, где $r \hm= \Rg A$. Это и есть ``то самое'' тождество.}:
    \[
      \boxed{\dim\Imag\phi + \dim\Ker\phi = \dim X}
    \]
    
    \medskip
    
    \textbf{Матрица отображения.} Пусть размерность $\mathcal L_1$ равна $l$, а размерность $\mathcal L_2$ равна $k$.
    Пусть $a \hm= (\bds a_1, \ldots, \bds a_l)$~---~базис в $\mathcal L_1$, а $b \hm= (\bds b_1, \ldots, \bds b_k)$~---~базис в $\mathcal L$.
    Тогда за базис в $\mathcal L$ предлагается взять $a \hm\cup b \hm= (\bds a_1, \ldots, \bds a_l, \bds b_1, \ldots, \bds b_k)$\footnote{Так можно сделать, потому что $\mathcal L \hm= \mathcal L_1 \hm\oplus \mathcal L_2$.}.
    
    В общем случае, столбцы матрицы отображения $\phi\colon X \hm\to Y$~---~это координатные столбцы базиса $X$ в базисе $Y$.
    В случае преобразования, $X$ и $Y$~---~одно и то же, и базис один.
    Поэтому столбцы матрицы преобразования $\phi$~---~это координатные столбцы образов базисных векторов в том же базисе (индексом $a \hm\cup b$ обозначено, в каком базисе компоненты)
    \[
      A = \bigl(\phi(\bds a_1)_{a \cup b}, \ldots, \phi(\bds a_l)_{a \cup b}, \phi(\bds b_1)_{a \cup b}, \ldots, \phi(\bds b_k)_{a \cup b}\bigr)
      = \begin{pmatrix}
        E_{l \times l} & 0_{l \times k}\\
        0_{k \times l} & 0_{k \times k}
      \end{pmatrix}
    \]
    так как $\phi(\bds a_i) \hm= 1 \hm\cdot \bds a_i$, а $\phi(\bds b_i) \hm= \bds 0$.
    
    Если же рассмотреть $\phi$ не как преобразование, а как отображение $\widetilde\phi\colon \mathcal L \hm\to \mathcal L_1$, то в данном случае базисы в пространствах ``из'' и ``куда'' уже отличаются.
    Столбцов в матрице отображения останется $l \hm+ k$, но строк уже будет всего $l$ (потому что базис в пространстве ``куда'' $\mathcal L_1$ есть $(\bds a_1, \ldots, \bds a_l)$).
    То есть матрица отображения $\phi\colon \mathcal L \hm\to \mathcal L_1$ равна (индексом $a$ обозначено, в каком базисе компоненты)
    \[
      \widetilde A = \bigl(\widetilde\phi(\bds a_1)_{a}, \ldots, \widetilde\phi(\bds a_l)_{a}, \widetilde\phi(\bds b_1)_{a}, \ldots, \widetilde\phi(\bds b_k)_{a}\bigr)
      = \begin{pmatrix}
        E_{l \times l} & 0_{l \times k}
      \end{pmatrix}
    \]
  \end{solution}
  
  
  \subsection{\# 23.29(5)}
  
  Линейное отображение $\phi\colon X \hm\to Y$ задано матрицей $A_{m\times n}$ в базисах $\bds e \hm= (\bds e_1, \ldots, \bds e_n)$ в $X$ и $\bds f \hm= (\bds f_1, \ldots, \bds f_m)$ в $Y$:
  \[
    A = \begin{pmatrix}
      -2 & -2 & 2\\
      -2 & -2 & 2\\
      -3 & -2 & 2\\
      4 & -1 & 1\\
      6 & 5 & -5
    \end{pmatrix}
  \]
  
  Надо найти ядро, множество значений отображения.
  Выяснить, является ли оно инъективным, сюръективным.
  
  \begin{solution}
    Из размера матрицы $A$ видно, что $\dim X \hm= 3$ и $\dim Y \hm= 5$.
    
    Найдём \textbf{ядро} $\phi$ (за $x \hm\in \RR^3$ обозначен координатный столбец вектора $\bds x \in X$):
    \[
      \bds x \in \Ker\phi \Leftrightarrow \phi(x) = Ax = 0
    \]
    
    Надо решить однородную систему, упростив матрицу $A$:
    \begin{equation}\label{p-23-29-gauss}
      \begin{pmatrix}
        -2 & -2 & 2\\
        -2 & -2 & 2\\
        -3 & -2 & 2\\
        4 & -1 & 1\\
        6 & 5 & -5
      \end{pmatrix}
      \sim \begin{pmatrix}
        \textcolor{violet}{\bds 1} & 1 & -1\\
        -2 & -2 & 2\\
        -3 & -2 & 2\\
        4 & -1 & 1\\
        6 & 5 & -5
      \end{pmatrix}
      \sim \begin{pmatrix}
        1 & 1 & -1\\
        0 & 0 & 0\\
        0 & \textcolor{violet}{\bds 1} & -1\\
        0 & -5 & 5\\
        0 & -1 & 1
      \end{pmatrix}
      \sim \begin{pmatrix}
        1 & 0 & 0\\
        0 & 0 & 0\\
        0 & 1 & -1\\
        0 & 0 & 0\\
        0 & 0 & 0
      \end{pmatrix}
    \end{equation}
    
    Сразу видно, что ранг матрицы отображения (он же ранг самого отображения) $\Rg\phi \hm= 2$.
    Ранг отображения~---~размерность множества значений.
    Поэтому $\dim\Imag\phi \hm= 2 \hm< 5 \hm= \dim Y$.
    Что означает, что отображение $\phi$ не сюръективно.
    
    Произвольный вектор из ядра представим как
    \[
      x = \begin{pmatrix}
        0 \\ t \\ t
      \end{pmatrix} = t \cdot \begin{pmatrix}
        0 \\ 1 \\ 1
      \end{pmatrix}
    \]
    
    Очевидно, что размерность ядра $\dim\Ker\phi \hm= 1$.
    Ядро не нулевое.
    Это значит, что отображение не инъективно (можно подобрать два различных вектора (отличающихся на ненулевой вектор из ядра) которые $\phi$ переводит в один и тот же вектор).
    
    \medskip
    
    Остаётся найти \textbf{множество значений} $\phi$ (у которого размерность, уже известно, равна двум):
    \[
      y \in \Imag\phi \Leftrightarrow \exists x \in X: Ax = y
    \]
    
    Получается, надо рассмотреть расширенную матрицу $(A \mid y)$, упростить её, и выписать ограничения на $y$, чтобы система была совместна.
    Упрощение матрицы $A$ уже проведено в (\ref{p-23-29-gauss}).
    Остаётся проделать те же самые преобразования со столбцом $y$:
    \[
      \begin{pmatrix}
        y_1 \\ y_2 \\ y_3 \\ y_4 \\ y_5
      \end{pmatrix}
      \sim \begin{pmatrix}
        -y_1/2 \\ y_2 \\ y_3 \\ y_4 \\ y_5
      \end{pmatrix}
      \sim \begin{pmatrix}
        -y_1/2 \\ y_2 - y_1 \\ y_3 - 3y_1/2 \\ y_4 + 4y_1/2 \\ y_5 + 6y_1/2
      \end{pmatrix}
      \sim \begin{pmatrix}
        -y_1/2 - \left(y_3 - 3y_1/2\right) \\ y_2 - y_1 \\ y_3 - 3y_1/2 \\ y_4 + 4y_1/2 + 5 \left(y_3 - 3y_1/2\right) \\ y_5 + 6y_1/2 + \left(y_3 - 3y_1/2\right)
      \end{pmatrix}
    \]
    
    Зануляем компоненты, соответствующие нулевым строкам в упрощённой $A$:
    \[
      \left\{
        \begin{aligned}
          &y_2 - y_1 = 0\\
          &y_4 + \frac{4y_1}{2} + 5 \left(y_3 - \frac{3y_1}{2}\right)\\
          &y_5 + \frac{6y_1}{2} + \left(y_3 - \frac{3y_1}{2}\right)
        \end{aligned}
      \right.
    \]
    
    Уравнений $3$ (очевидно, линейно независимых).
    Переменных $5$.
    Значит, какие-то три можно выразить через другие две.
    Например,
    \[
      \left\{
        \begin{aligned}
          &y_2 = y_1\\
          &y_4 = \frac{1}{2} (11y_1 - 10y_3)\\
          &y_5 = \frac{1}{2} (-3y_1 - 2y_3)
        \end{aligned}
      \right.
    \]
    
    Поэтому общий вид $y$ из множества значений ($y_1 \hm= t_1 \hm\in \RR$, $y_3 \hm= t_2 \hm\in \RR$):
    \[
      y = \begin{pmatrix}
        t_1 \\ t_1 \\ t_2 \\ \frac{1}{2} (11 t_1 - 10 t_2) \\ \frac{1}{2} (-3 t_1 - 2 t_2)
      \end{pmatrix}
      = \underbrace{2t_1}_{\tilde{t}_1} \cdot \begin{pmatrix}
        2 \\ 2 \\ 0 \\ 11 \\ -3
      \end{pmatrix} + \underbrace{2t_2}_{\tilde{t}_2} \begin{pmatrix}
        0 \\ 0 \\ 1 \\ -5 \\ -1
      \end{pmatrix}
    \]
    
    Базис в $\Imag\phi$ ``виден''.
    Векторов в нём, ожидаемо, два...
    
    \medskip
    
    \emph{Другой способ нахождения $\Imag \phi$.}
    Видно, что в матрице $A$ первые два столбца линейно независимы, а третий выражается через второй.
    Поэтому $\Imag \phi$~---~это линейная оболочка первых двух столбцов матрицы $A$.
  \end{solution}
  
  
  \subsection{\# 23.40(1в)}
  
  Пусть $\mathcal P^{(m)}$~---~линейное пространство линейных многочленов степени не выше $m$.
  Проверить, что дифференцирование, рассматриваемое как преобразование $D\colon \mathcal P^{(m)} \hm\to \mathcal P^{(m)}$, линейно.
  Найти его ядро и множество значений.
  Вычислить матрицу в базисе
  \[
    \left(1, \frac{t}{1!}, \ldots, \frac{t^m}{m!}\right)
  \]
  
  \begin{solution}
    \textbf{Линейность} ($p(t) \hm= a_0 \hm+ a_1 t \hm+ \ldots \hm+ a_m t^m$ и $q(t) \hm= b_0 \hm+ b_1 t \hm+ \ldots \hm+ b_m t^m$):
    \begin{equation*}
    \begin{split}
      D(p(t) + q(t)) = &D\bigl((a_0 + a_1 t + \ldots + a_m t^m) + (b_0 + b_1 t + \ldots b_m t^m)\bigr)\\
      = &a_1 + \ldots + a_m t^{m - 1} + b_1 + \ldots + b_m t^{m - 1}
      = D(p(t)) + D(q(t))
    \end{split}
    \end{equation*}
    
    Аналогично с $D(\alpha p(t)) \hm= \alpha D(t(t))$.
    
    \medskip
    
    \textbf{Ядро}:
    \[
      p(t) \in \Ker D \Leftrightarrow D(p(t)) = a_1 + \ldots + a_m t^{m - 1} = 0 \Leftrightarrow a_1 = \ldots = a_m = 0
    \]
    
    Поэтому $p(t) \in \Ker D \hm\Leftrightarrow p(t) \hm= a_0$.
    То есть ядро дифференцирования~---~все константные многочлены.
    Размерность ядра, очевидно, равно одному.
    
    \medskip
    
    \textbf{Множество значений}~---~это $\mathcal P^{(m - 1)}$.
    Потому что, с одной стороны, при дифференцировании многочлена степени не выше $m$ получается многочлен степени не выше $m \hm- 1$.
    С другой стороны, для многочлена $h(t)$ степени не выше $m \hm- 1$ можно подобрать многочлен $p(t)$ степени не выше $m$, дифференцирование которого даёт данный (например, $p(t) \hm= \int h(t)$).
    
    \medskip
    
    \textbf{Матрица преобразования.}
    Стобцы матрицы~---~координаты образов векторов пространства $\mathcal P^{(m)}$ в том же базисе.
    Так как
    \[
      \begin{aligned}
        &D\left(1\right) = 0\\
        &D\left(\frac{1}{1!} t\right) = \textcolor{green}{\bds 1} \cdot \frac{1}{1!}\\
        &D\left(\frac{1}{2!} t^2\right) = \textcolor{orange}{\bds 1} \cdot \frac{1}{1!} t\\
        &\ldots\\
        &D\left(\frac{1}{m!} t^m\right) = \textcolor{blue}{\bds 1} \cdot \frac{1}{(m - 1)!} t^{m - 1}
      \end{aligned}
    \]
    то матрица преобразования получается равной
    \[
      A = \begin{pmatrix}
        0      & \textcolor{green}{\bds 1} & 0                          & \ldots & 0\\
        0      & 0                         & \textcolor{orange}{\bds 1} & \ldots & 0\\
        \vdots & \vdots                    & \vdots                     & \ddots & \vdots\\
        0      & 0                         & 0                          & 0      & \textcolor{blue}{\bds 1}\\
        0      & 0                         & 0                          & 0      & 0
      \end{pmatrix}
    \]
    
    Размер матрицы $A$ равен $(m \hm+ 1) \hm\times (m \hm+ 1)$.
    
    Если бы дифференцирование рассматривалось как отображение $D \colon \mathcal P^{(m)} \hm\to \mathcal P^{(m - 1)}$ (и в $\mathcal P^{(m - 1)}$ базис бы был такой же, как в $\mathcal P^{(m)}$, только без последнего вектора), то строк в матрице отображения стало бы меньше на одну (столбцов столько же, потому что базис исходного пространства такой же, строк же меньше, потому что меняется базис пространства, куда отображаем):
    \[
      A' = \begin{pmatrix}
        0      & \textcolor{green}{\bds 1} & 0                          & \ldots & 0\\
        0      & 0                         & \textcolor{orange}{\bds 1} & \ldots & 0\\
        \vdots & \vdots                    & \vdots                     & \ddots & \vdots\\
        0      & 0                         & 0                          & 0      & \textcolor{blue}{\bds 1}
      \end{pmatrix}
    \]
    
    И размер матрицы $A'$ равен $m \hm\times (m \hm+ 1)$.
  \end{solution}
  
  
  \subsection{\# 23.82(1)}
  
  Преобразование $\phi$ переводит линейно независимые векторы $a_i$ в $b_i$.
  Преобразование $\psi$ переводит линейно независимые векторы $b_i$ в $c_i$.
  То есть
  \[
    \begin{aligned}
      &(a_1, a_2, a_3) \xrightarrow{\phi} (b_1, b_2, b_3)\\
      &(b_1, b_2, b_3) \xrightarrow{\psi} (c_1, c_2, c_3)
    \end{aligned}
  \]
  
  Надо найти матрицу $A_{\psi\phi}$ преобразования $\psi\phi$ в исходном базисе.
  
  \begin{solution}
    Пусть матрицы $A_{\phi}$ и $A_{\psi}$~---~матрицы преобразований $\phi$ и $\psi$ соответсвенно.
    Тогда условие задачи можно переписать в виде
    \[
      \left\{
        \begin{aligned}
          &A_{\phi} (a_1, a_2, a_3) = (b_1, b_2, b_3)\\
          &A_{\psi} (b_1, b_2, b_3) = (c_1, c_2, c_3)\\
          &A_{\psi\phi} (a_1, a_2, a_3) = (c_1, c_2, c_3)
        \end{aligned}
      \right.
    \]
    
    Если положить $A \hm\equiv (a_1, a_2, a_3)$, $B \hm\equiv (b_1, b_2, b_3)$ и $C \hm\equiv (c_1, c_2, c_3)$, то условие можно переписать в ещё более сжатой форме:
    \[
      \left\{
        \begin{aligned}
          &A_{\phi} A = B\\
          &A_{\psi} B = C\\
          &A_{\psi\phi} A = C
        \end{aligned}
      \right.
    \]
    
    Отсюда (так как по условию существуют $A^{-1}$ и $B^{-1}$):
    \[
      \left\{
        \begin{aligned}
          &A_{\phi} = B A^{-1}\\
          &A_{\psi} = C B^{-1}\\
          &A_{\psi \phi} = C A^{-1} = CB^{-1} \cdot B A^{-1} = A_{\psi} A_{\phi}
        \end{aligned}
      \right.
    \]
    
    А как бы, например, выглядела матрица преобразования $\phi$ в базисе $(a_1, a_2, a_3)$?
    На данном этапе известно, что
    \[
      A_{\psi\phi} \xi_{\bds e} = \nu_{\bds e}
    \]
    где индекс $\bds e$ показывает, в каком базисе даны компоненты.
    При замене старого базиса $\bds e \hm= (e_1, e_2, e_3)$ на новый $\bds a \hm= (a_1, a_2, a_3)$ векторы базисов связаны соотношением (коэффициенты разложения векторов нового базиса $\bds a$ по старому $\bds e$ образуют столбцы матрицы перехода $S$):
    \[
      (a_1, a_2, a_3) = (e_1, e_2, e_3) S
    \]
    
    Координатные же столбцы произвольного вектора $\bds x$ в старом $\xi_{\bds e}$ и новом $\xi_{\bds a}$ базисах связаны следующим образом:
    \[
      \left\{
        \begin{aligned}
          &\bds x = \bds e \xi_{\bds e}\\
          &\bds x = \bds a \xi_{\bds a} = \bds e S \xi_{\bds a}\\
        \end{aligned}
        \Rightarrow \xi_{\bds e} = S \xi_{\bds a}
      \right.
    \]
    
    Итак, пока известно, что в базисе $\bds e$:
    \begin{equation}\label{p-23-82-old-A}
      A_{\psi \phi} \xi_{\bds e} = \nu_{\bds e}
    \end{equation}
    
    Надо же найти матрицу $A'_{\psi\phi}$ преобразования в базисе $\bds a$:
    \begin{equation}\label{p-23-82-new-A}
      A'_{\psi\phi} \xi_{\bds a} = \nu_{\bds a}
    \end{equation}
    
    Чтобы найти $A'_{\psi\phi}$, можно выразить в формуле (\ref{p-23-82-old-A}) координатные столбцы в старом базисе через столбцы соответствующих векторов в новом базисе (чтобы получить формулу, ``похожую'' на (\ref{p-23-82-new-A}), только с $A_{\psi\phi}$ вместо $A'_{\psi\phi}$):
    \begin{equation}\label{p-23-82-new-A-like-old}
      A_{\psi\phi} S \xi_{\bds a} = S \nu_{\bds a}
      \Leftrightarrow S^{-1} A_{\psi\phi} S \xi_{\bds a} = \nu_{\bds a}
    \end{equation}
    
    Подставляя вместо $\xi_{\bds a}$ базисные столбцы в последнюю формулу (\ref{p-23-82-new-A-like-old}) и в (\ref{p-23-82-new-A}) получаем, что
    \[
      S^{-1} A_{\psi\phi} S = A'_{\psi\phi}
    \]
    
    Остаётся понять, чему равна матрица $S$ перехода от базиса $\bds e$ к базису $\bds a$.
    Столбцы матрицы $A$~---~координаты новых базисных векторов $\bds a$ в старом базисе $\bds e$.
    Поэтому $A$ и есть $S$, и матрица преобразования в новом базисе в итоге выглядит так:
    \[
      A'_{\psi\phi} = A^{-1} A_{\psi\phi} A
    \]
  \end{solution}
  
  
  \subsection{\# 31.21}
  Пусть $t_0 \in \RR$~---~фиксированное число.
  
  Показать, что сопоставление $f$ каждому многочлену степени не выше $n$ его значения в $t_0$ есть линейная функция:
  \[
    f\colon \mathcal P^{(n)} \ni p(t) \mapsto p(t_0) \in \RR
  \]
  
  Вычислить координатную строку функции в базисах $(1, t, \ldots, t^n)$ и $(1, t \hm- t_0, \ldots, (t \hm- t_0)^n)$.
  
  \begin{solution}
    Линейность:
    \begin{equation*}
    \begin{split}
      f\bigl(p(t) + q(t)\bigr) &= f\bigl((a_{0p} + a_{1p} t + \ldots + a_{np} t^n) + (a_{0q} + a_{1q} t + \ldots + a_{nq} t^n)\bigr)\\
      &= (a_{0p} + a_{1p} t_0 + \ldots + a_{np} t_0^n) + (a_{0q} + a_{1q} t_0 + \ldots + a_{nq} t_0^n)
      = f\bigl(p(t)\bigr) + f\bigl(q(t)\bigr)
    \end{split}
    \end{equation*}
    
    Аналогично и
    \[
      f\bigl(\alpha p(t)\bigr) = f(\alpha a_{0p} + \alpha a_{1p} t + \ldots + \alpha a_{np} t^n)
      = \alpha a_{0p} + \alpha a_{1p} t_0 + \ldots + \alpha a_{np} t_0^n
      = \alpha f\bigl(p(t)\bigr)
    \]
    
    Координатная строка функции~---~как матрица отображения, только в случае с функцией $\phi\colon X \hm\to \RR$ она становится строкой.
    То есть надо вычислить образы базисных векторов и разложить их ``по базису в $\RR$'' (который принимается равным просто $1$).
    
    Итак, в случае, если в $\mathcal P^{(n)}$ выбран базис $(1, t, \ldots, t^n)$, то координатная строка функции $f$ равна:
    \[
      \bigl(f(1), f(t), \ldots, f(t^n)\bigr) = (1, t_0, \ldots, t_0^n)
    \]
    
    Если же в $\mathcal P^{(n)}$ выбран базис $(1, t \hm- t_0, \ldots, (t \hm- t_0)^n)$, то координатная строка функции $f$ будет равна:
    \[
      \bigl(f(1), f(t), \ldots, f(t^n)\bigr) = (1, 0, \ldots, 0)
    \]
  \end{solution}
  
  
  \subsection{\# 31.35(1)}
  
  Пусть базису $(\bds e_1, \bds e_2, \bds e_3)$ пространства $\mathcal L$ биортогонален базис $(f_1, f_2, f_3)$ сопряжённого пространства $\mathcal L^*$.
  Надо найти базис, биортогональный базису
  \[
    \bds e_1' = \bds e_1 + \bds e_2,
    \quad \bds e_2' = \bds e_2 + \bds e_3,
    \quad \bds e_3' = \bds e_3
  \]
  
  \begin{solution}
    Биортогональный базис $f$~---~это такой базис в $\mathcal L^*$, функции $f_i$ которого обладают следующим свойством:
    \[
      f_i(\bds e_j) = \left\{
        \begin{aligned}
          &1, \quad \mbox{если $i = j$}\\
          &0, \quad \mbox{если $ i\not= j$}
        \end{aligned}
      \right.
    \]
    
    То есть для неизвестных пока линейных функций $f_1', f_2', f_3'$ нового биортогонального базиса уже известно, что, во-первых (выписываем соотношения для $f_1'$):
    \[
      \left\{
        \begin{aligned}
          &1 = f_1'(\bds e_1') = f_1'(\bds e_1 + \bds e_2) = f_1'(\bds e_1) + f_1'(\bds e_2)\\
          &0 = f_1'(\bds e_2') = f_1'(\bds e_2 + \bds e_3) = f_1'(\bds e_2) + f_1'(\bds e_3)\\
          &0 = f_1'(\bds e_3') = f_1'(\bds e_3)
        \end{aligned}
      \right.
    \]
    То есть $f_1'$ в точности такая же, как $f_1$ (ведь линейная функция из $\mathcal L^*$ однозначно определяется значениями на векторах базиса $\mathcal L$).
    
    Далее, аналогичные соотношения для $f_2'$:
    \[
      \left\{
        \begin{aligned}
          &0 = f_2'(\bds e_1') = f_2'(\bds e_1 + \bds e_2) = f_2'(\bds e_1) + f_2'(\bds e_2)\\
          &1 = f_2'(\bds e_2') = f_2'(\bds e_2 + \bds e_3) = f_2'(\bds e_2) + f_2'(\bds e_3)\\
          &0 = f_2'(\bds e_3') = f_2'(\bds e_3)
        \end{aligned}
      \right.
      \Leftrightarrow
      \left\{
        \begin{aligned}
          &f_2'(\bds e_1) = -1\\
          &f_2'(\bds e_2) = 1\\
          &f_2'(\bds e_3) = 0
        \end{aligned}
      \right.
    \]
    
    И для $f_3'$:
    \[
      \left\{
        \begin{aligned}
          &0 = f_3'(\bds e_1') = f_3'(\bds e_1 + \bds e_2) = f_3'(\bds e_1) + f_3'(\bds e_2)\\
          &0 = f_3'(\bds e_2') = f_3'(\bds e_2 + \bds e_3) = f_3'(\bds e_2) + f_3'(\bds e_3)\\
          &1 = f_3'(\bds e_3') = f_3'(\bds e_3)
        \end{aligned}
      \right.
      \Leftrightarrow
      \left\{
        \begin{aligned}
          &f_3'(\bds e_1) = 1\\
          &f_3'(\bds e_2) = -1\\
          &f_3'(\bds e_3) = 1
        \end{aligned}
      \right.
    \]
    
    Чему равны $f_2'$ и $f_3'$?
    Линейные действия над функциями из $\mathcal L^*$ равносильны таким же линейным действиям над их координатными строками: пространство $\mathcal L^*$ изоморфно пространству строк размера $\dim\mathcal L$.
    Функции $f_2'$, например, (при базисе $\bds e$ в $\mathcal L$) соответствует строка $(-1, 1, 0)$.
    Можно разложить эту строку в линейную комбинацию базисных строк (соответствующих функциям, из которого состоит биортогональный базис $f$):
    \[
      (-1, 1, 0) = -1 \cdot (1, 0, 0) + 1 \cdot (0, 1, 0)
    \]
    Это значит, что с самой функцией $f_2'$ будет ``то же самое'':
    \[
      f_2' = -1 \cdot f_1 + 1 \cdot f_2
    \]
    
    Аналогично с $f_3'$:
    \[
      f_3' = 1 \cdot f_1 - 1 \cdot f_2 + 1 \cdot f_3
    \]
    
    Итого, новый биортогональный базис:
    \[
      f = (f_1, -f_1 + f_2, f_1 - f_2 + f_3)
    \]
  \end{solution}
\end{document}
