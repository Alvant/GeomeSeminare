\documentclass[a4paper,12pt]{article}

\usepackage{mystyle}
\usepackage{gensymb}


\usepackage{scalerel}
\usepackage{stackengine}

\graphicspath{ {images/} }


\definecolor{violet}{RGB}{148, 0, 211}
\definecolor{pink}{RGB}{218, 3, 174}



% https://tex.stackexchange.com/a/101138/135045

\newcommand\widesim[1]{\ThisStyle{%
  \setbox0=\hbox{$\SavedStyle#1$}%
  \stackengine{-.1\LMpt}{$\SavedStyle#1$}{%
    \stretchto{\scaleto{\SavedStyle\mkern.2mu\sim}{.5150\wd0}}{.6\ht0}%
  }{O}{c}{F}{T}{S}%
}}

\newcommand{\BigMiddleThree}{\;\left|\vphantom{\begin{pmatrix} 0\\0\\0 \end{pmatrix}}\right.\;}
\newcommand{\BigMiddleFour}{\;\left|\vphantom{\begin{pmatrix} 0\\0\\0\\0 \end{pmatrix}}\right.\;}


\newcommand{\Coso}{\mathcal S'}


% https://tex.stackexchange.com/questions/63531/how-to-write-quotation-marks-in-math-environment
\DeclareMathSymbol{\mlq}{\mathord}{operators}{``}
\DeclareMathSymbol{\mrq}{\mathord}{operators}{`'}



\author{Алексеев Василий}


\title{Семинар 3}
\date{17 февраля + 21 февраля 2023}


\begin{document}
  \maketitle
  
  \tableofcontents

  \thispagestyle{empty}
  
  \newpage
  
  \pagenumbering{arabic}


  \section{Линейные пространства $1$}
  
  \subsection{Кососимметричность, или ``Плоскости, плоскости повсюду''}
  
  Возьмём \emph{множество} квадратных матриц $\RR^{3 \times 3} \hm\equiv \mathcal M$.
  И рассмотрим в нём \emph{подмножество} $\mathcal S'$ кососимметрических\footnote{Множество ``без штриха'' $\mathcal S \hm\subset \mathcal M$ как бы означает совокупность всех \emph{симметрических} матриц, и тогда ``$\mathcal S$ штрихованная''~---~кососимметрические. Может возникнуть вопрос: зачем вообще было ``штриховать'', ведь можно было просто работать с симметрическими матрицами~$\mathcal S$ (какая по сути разница)~---~и не пришлось бы так сложно пояснять обозначения?.. Ясного ответа на этот вопрос нет. Так уж ``сложилось''.} матриц:
  \[
    \Coso = \{A \in \mathcal M \mid A^T = -A\}
  \]
  
  Что можно ``заметить'' про множество~$\Coso$?
  Будет ли, например, кососимметричной \emph{сумма двух матриц} $A, B \hm\in \Coso$?
  \[
    (A + B)^T = A^T + B^T = (-A) + (-B) = -(A + B)
  \]
  То есть, да, сумма также кососимметрична $A \hm+ B \hm\in \Coso$.
  Также несложно убедиться, что кососимметричность сохранится и при \emph{умножении матрицы на число}.
  То есть $\alpha A \hm\in \Coso$ при $A \hm\in \Coso$ и $\alpha \hm\in \RR$.
  В таком случае про множество кососимметрических матриц $\Coso$ можно сказать, что оно \emph{замкнуто} относительно операций сложения и умножения на число.
  
  С подобными замкнутыми подмножествами мы уже встречались: например, плоскость как \emph{подпространство} векторов (направленных отрезков) замкнуто во всём геометрическом \emph{пространстве} векторов.
  То есть о кососимметричных матрицах $\Coso$ можно думать как о ``плоскости'' (в некотором смысле) в \emph{пространстве} всех квадратных матриц~$\mathcal M$.
  
  Возвращаясь к ``векторам из аналитической геометрии'': плоскость~---~\emph{двумерное} подпространство.
  Всё же пространство~---~\emph{трёхмерное}.
  Это значит, что, например, во всём пространстве можно выбрать \emph{базис} из трёх векторов, то есть систему из трёх упорядоченных линейно независимых векторов, линейной комбинацией которых может быть выражен любой вектор пространства.
  (На плоскости~---~можно выбрать базис из двух векторов.)
  Какой будет \emph{размерность} пространства матриц~$\mathcal M$?
  Возьмём произвольную матрицу~$A$ из~$\mathcal M$:
  \[
    A = \begin{pmatrix}
      a_{11} & a_{12} & a_{13}\\
      a_{21} & a_{22} & a_{23}\\
      a_{31} & a_{32} & a_{33}
    \end{pmatrix}
  \]
  Она однозначно определяется своими $3 \hm\times 3 \hm= 9$ элементами.
  Таким образом... размерность $\mathcal M$ равна $9$?
  Можно ли выбрать базис (полную систему линейно независимых \emph{матриц}) в $\mathcal M$ в количестве $9$ штук?
  Ещё раз посмотрим на произвольную матрицу $A \hm\in \mathcal M$ и просто представим её (довольно очевидным, если подумать, образом) как сумму:
  \begin{equation*}
  \begin{split}
    A = \begin{pmatrix}
      a_{11} & a_{12} & a_{13}\\
      a_{21} & a_{22} & a_{23}\\
      a_{31} & a_{32} & a_{33}
    \end{pmatrix}
    &= a_{11} \overbrace{\begin{pmatrix}
      1 & 0 & 0\\
      0 & 0 & 0\\
      0 & 0 & 0
    \end{pmatrix}}^{E_1}
    + a_{12} \overbrace{\begin{pmatrix}
      0 & 1 & 0\\
      0 & 0 & 0\\
      0 & 0 & 0
    \end{pmatrix}}^{E_2}
    + a_{13} \overbrace{\begin{pmatrix}
      0 & 0 & 1\\
      0 & 0 & 0\\
      0 & 0 & 0
    \end{pmatrix}}^{E_3}\\
    %%
    &+ a_{21} \overbrace{\begin{pmatrix}
      0 & 0 & 0\\
      1 & 0 & 0\\
      0 & 0 & 0
    \end{pmatrix}}^{E_4}
    + a_{22} \overbrace{\begin{pmatrix}
      0 & 0 & 0\\
      0 & 1 & 0\\
      0 & 0 & 0
    \end{pmatrix}}^{E_5}
    + a_{23} \overbrace{\begin{pmatrix}
      0 & 0 & 0\\
      0 & 0 & 1\\
      0 & 0 & 0
    \end{pmatrix}}^{E_6}\\
    %%
    &+ a_{31} \overbrace{\begin{pmatrix}
      0 & 0 & 0\\
      0 & 0 & 0\\
      1 & 0 & 0
    \end{pmatrix}}^{E_7}
    + a_{32} \overbrace{\begin{pmatrix}
      0 & 0 & 0\\
      0 & 0 & 0\\
      0 & 1 & 0
    \end{pmatrix}}^{E_8}
    + a_{33} \overbrace{\begin{pmatrix}
      0 & 0 & 0\\
      0 & 0 & 0\\
      0 & 0 & 1
    \end{pmatrix}}^{E_9}
  \end{split}
  \end{equation*}
  
  Произвольная матрица разложена~---~значит, система матриц $\{E_1, \ldots, E_9\}$ полная.
  Очевидно, она также линейно независимая (несложно проверить, что только тривиальная, то есть с нулевыми коэффициентами, линейная комбинация матриц даст нулевую матрицу).
  Таким образом, в качестве базиса в $\mathcal M$ можно выбрать указанную систему из $9$ матриц~---~размерность $\mathcal M$ в самом деле равна $9$.
  
  А какова будет размерность подпространства кососимметрических матриц $\Coso$?
  Пусть $C \hm\in \Coso$.
  Что можно сказать про~$C$?
  Очевидно, она уже определяется не девятью элементами~---~ведь симметричные относительно главной диагонали отличаются лишь знаками.
  Несложно прийти к выводу, что произольную матрицу из $\Coso$ можно представить, например, в таком виде:
  \[
    C = \begin{pmatrix}
       0 &  a & b\\
      -a &  0 & c\\
      -b & -c & 0
    \end{pmatrix}, \quad a, b, c \in \RR
  \]
  
  То есть любая $C \hm\in \Coso$ однозначно задаётся лишь \emph{тремя} числами $(a, b, c)$.
  Размерность $\Coso$ равна $3$?
  Да, снова несложно выбрать базис из нужного числа матриц (кососимметрических матриц!):
  \[
    C = \begin{pmatrix}
       0 &  a & b\\
      -a &  0 & c\\
      -b & -c & 0
    \end{pmatrix}
    = a \overbrace{\begin{pmatrix}
       0 &  1 & 0\\
      -1 &  0 & 0\\
       0 &  0 & 0
    \end{pmatrix}}^{C_1}
    + b \overbrace{\begin{pmatrix}
       0 &  0 & 1\\
       0 &  0 & 0\\
      -1 &  0 & 0
    \end{pmatrix}}^{C_2}
    + c \overbrace{\begin{pmatrix}
       0 &  0 & 0\\
       0 &  0 & 1\\
       0 & -1 & 0
    \end{pmatrix}}^{C_3}
  \]
  
  Матрицы $\{C_1, C_2, C_3\}$ дают базис в $\Coso$.
  Значит, размерность $\Coso$ в самом деле равна трём.
  Зная базис, мы также можем представить $\Coso$ в таком виде:
  \[
    \Coso = \boxed{\{\alpha_1 C_1 + \alpha_2 C_2 + \alpha_3 C_3 \mid \alpha_1, \alpha_2, \alpha_3 \in \RR\} = \mathcal L(C_1, C_2, C_3)}
  \]
  то есть как \emph{линейную оболочку} $\mathcal L(C_1, C_2, C_3)$ системы матриц $\{C_1, C_2, C_3\}$.
  
  Итак, выбор базиса, например $\{C_1, C_2, C_3\}$, в $\Coso$ позволяет установить \emph{взаимно однозначное соответствие} (биекцию) между матрицами $\Coso$ и вектор-столбцами $\RR^3$ из \emph{координат} матриц в базисе:
  \[
    \Coso \ni C = \begin{pmatrix}
       0 &  a & b\\
      -a &  0 & c\\
      -b & -c & 0
    \end{pmatrix} \longleftrightarrow \begin{pmatrix}
      a \\ b \\ c
    \end{pmatrix} = \xi \in \RR^3
  \]
  
  Но это не просто взаимно однозначное соответствие...
  Пусть есть две матрицы из $\Coso$: матрица $A_1$, которой соответствует столбец координат $\xi_1 \hm= (a_1, b_1, c_1)^T$, и $A_2$ со столбцом $\xi_2 \hm= (a_2, b_2, c_2)^T$.
  Их сумма:
  \[
    \Coso \ni A_1 + A_2 = \begin{pmatrix}
       0           &  a_1 + a_2   & b_1 + b_2\\
      -(a_1 + a_2) &  0           & c_1 + c_2\\
      -(b_1 + b_2) & -(c_1 + c_2) & 0
    \end{pmatrix} \leftrightarrow \begin{pmatrix}
      a_1 + a_2\\
      b_1 + b_2\\
      c_1 + c_2
    \end{pmatrix} = \xi_1 + \xi_2 \in \RR^3
  \]
  
  То есть сумме матриц соответствует координатный столбец, являющийся суммой координатных столбцов матриц-слагаемых.
  Аналогично, результат умножения матрицы~$A$ на число~$\alpha$ есть матрица~$\alpha A$, координатный столбец которой получен умножением на то же число~$\alpha$ координатного столбца исходной матрицы~$A$.
  В таком случае можно сказать, что отображение между матрицами и координатными столбцами, помимо того, что оно взаимно однозначное, ещё и \emph{сохраняет линейные операции} (сумму и умножение на число), то есть является \emph{изоморфизмом}.
  
  
  \subsection{Линейные пространства}
  
  Дадим общие определения некоторым уже использовавшимся ранее, но так пока по-нормальному и не введённым, понятиям.
  
  Вещественным \emph{линейным пространством} называется \emph{множество объектов} $\mathcal V$, называемых векторами, для которых определена \emph{операция сложения} $\mlq{+}\mrq\colon \mathcal V \hm\times \mathcal V \hm\to \mathcal V$ и умножения на действительное число $\mlq{\cdot}\mrq\colon \RR \hm\times \mathcal V \hm\to \mathcal V$, удовлетворяющие следующим свойствам ($\bds u, \bds v, \bds w \hm\in \mathcal V$, $\alpha, \beta \hm\in \RR$)\footnote{Свойства ``очень похожи'' на свойства аналогичных операций на \href{https://github.com/Alvant/GeomeSeminare/blob/master2022/seminars/geome/seminar01}{множестве матриц} и на \href{https://github.com/Alvant/GeomeSeminare/tree/master2022/seminars/geome/seminar02}{множестве векторов}...}:
  \begin{enumerate}
    \item $\bds u + \bds v = \bds v + \bds u$
    
    (коммутативность сложения)
    
    \item $(\bds u + \bds v) + \bds w = \bds u + (\bds v + \bds w)$
    
    (ассоциативность сложения)
    
    \item $\exists \bds 0 \in \mathcal V: \bds v + \bds 0 = \bds v$
    
    (существование ``нулевого''~---~нейтрального относительно сложения~---~вектора)
    
    \item $\forall \bds v\ \exists (-{\bds v}): \bds v + (-{\bds v}) = \bds 0$
    
    (существование ``противоположного''~---~обратного по сложению~---~вектора)
    
    \item $(\alpha \beta) \bds v = \alpha (\beta \bds v)$
    \item $1 \cdot \bds v = \bds v$
    \item $\alpha (\bds u + \bds v) = \alpha \bds u + \alpha \bds v$
    
    (дистрибутивность умножения относительно сложения векторов)
    
    \item $(\alpha + \beta) \bds v = \alpha \bds v + \beta \bds v$
    
    (дистрибутивность умножения относительно сложения чисел)
  \end{enumerate}
  
  % TODO: definition breaks on two pages
  \newpage
  
  Множество $\mathcal V'$ называется \emph{подпространством} линейного пространства $\mathcal V$, если
  \begin{itemize}
    \item $\mathcal V'$ само является линейным пространством (с теми же операциями сложения и умножения на число, что для векторов $\mathcal V$\footnote{На самом деле это, скорее, \emph{ограничения} операций~$\mlq{+}\mrq$ и~$\mlq{\cdot}\mrq$ на подмножество $\mathcal V'$. Так, если ``обычная'' операция сложения $\mlq{+}\mrq$ определена как отображение $\mathcal V \hm\times \mathcal V \hm\to \mathcal V$, то её ограничение на $\mathcal V'$ определено как ``точно такое же'' сложение, только отображающее $\mathcal V' \hm\times \mathcal V' \hm\to \mathcal V'$, то есть ``работающее'' только на парах элементов из $\mathcal V'$.})
    \item векторы $\mathcal V'$ являются подмножеством векторов $\mathcal V$ ($\mathcal V' \subseteq \mathcal V$)
  \end{itemize}
  
  Чтобы проверить, является или нет данное подмножество $\mathcal V' \hm\subseteq \mathcal V$ векторов подпространством, можно, получается, просто проверить выполнимость свойств, которым должны удовлетворять операции сложения и умножения на число.
  Но так как операции на векторах предполагаемого подпространства $\mathcal V'$ по сути \emph{те же самые}, что и на векторах всего пространства $\mathcal V$, то проверять по-честному все свойства не обязательно (они ведь точно выполнены для векторов $\mathcal V$).
  Надо лишь проверить, что $\mathcal V'$ \emph{замкнуто} относительно этих операций, то есть что сумма произвольных двух векторов из $\mathcal V'$ лежит там же, как и произведение произвольного вектора из $\mathcal V'$ на любое число\footnote{Иными словами, надо фактически проверить корректность ограничения операций~$\mlq{+}\mrq$ и~$\mlq{\cdot}\mrq$ на подмножество $\mathcal V'$.}\textsuperscript{,}\footnote{Если $\mathcal V'$ замкнуто относительно умножения на число, то $\bds 0 \hm\in \mathcal V'$, так как (можно показать) $\bds 0 \hm= 0 \hm\cdot \bds v$, и также $-{\bds v} \hm\in \mathcal V'$, потому что $-{\bds v} \hm= -1 \hm\cdot \bds v$ ($\forall \bds v \hm\in \mathcal V'$).}.
  
  \emph{Базисом} в пространстве $\mathcal V$ называется упорядоченная, полная, линейно независимая система векторов $\{\bds e_1, \ldots, \bds e_n\}$.
  Во всех базисах одного и того же пространства $\mathcal V$ одинаковое число векторов $n$, которое называется \emph{размерностью пространства} и обозначается $\dim \mathcal V$.
  
  С абстрактными векторами (``вектора из линейной алгебры'') точно так же, как и с векторами~--~направленными отрезками (``вектора из аналитической геометрии''), работает переход между базисами.
  Так, пусть в $\mathcal V$ есть базис $e \hm= (\bds e_1, \ldots, \bds e_n)$ (``старый'', вектора базиса собраны в строчку длины $n$) и базис $e' \hm= (\bds e_1', \ldots, \bds e_n')$ (``новый'').
  Причём известно, как векторы $e'$ раскладываются по векторам $e$, то есть $e' \hm= e S$, где $S$~---~матрица перехода от ``старого'' базиса к ``новому'' (в её столбцах собраны компоненты ``новых'' базисных векторов в ``старом'' базисе).
  Получим тогда связь между координатным столбцом $\xi \hm\in \RR^n$ произвольного вектора $\bds x \hm\in \mathcal V$ в базисе $e$ и его же координатным столбцом $\xi' \hm\in \RR^n$ в базисе $e'$:
  \[
    e \xi = \bds x = e' \xi' = eS \cdot \xi' \Rightarrow e (\xi - S \xi') = \bds 0 \Rightarrow \boxed{\xi = S \xi'}
  \]
  
  

  \section{Задачи}
  
  
  \subsection{\# 20.3(1, 4)}
  
  Выяснить, являются ли линейными подпространствами следующие множества векторов в $n$-мерном пространстве:
  множество векторов $\mathcal V_1$, все координаты которых равны между собой?
  множество векторов $\mathcal V_2$, сумма координат которых равна $1$?
  
  \begin{solution}
    Пусть $\bds a, \bds b \in \mathcal V_1$.
    Очевидно, у вектора-суммы $\bds a \hm+ \bds b$ все координаты также одинаковые.
    При умножении вектора $\bds a$ на произвольное число $\alpha \hm\in \RR$ также получаем вектор из $\mathcal V_1$.
    То есть $\mathcal V_1$ замкнуто относительно сложения и умножения, и потому является подпространством.
    Размерность этого подпространства равна единице (``прямая''), так как есть базис $\{\bds e_1\}$ из одного вектора, например $\bds e_1 \hm= (1, \ldots, 1)^T$, состоящего из всех единиц\footnote{Как часто уже было и ещё будет, равенство между вектором и столбцом из компонент в данном случае~---~это не совсем ``равенство''. Это значит, что множество векторов изоморфно множеству столбцов, состоящих из компонент векторов в фиксированном базисе. Поэтому вектор можно отождествлять с его координатным столбцом. При этом собственно настоящим вектором на самом деле может быть ``что угодно'' (направленный отрезок, матрица, функция, ...)}:
    \[
      \mathcal V_1 \ni \bds a = \begin{pmatrix}
        a \\ \vdots \\ a
      \end{pmatrix} = a \cdot \begin{pmatrix}
        1 \\ \vdots \\ 1
      \end{pmatrix}
    \]
    
    \medskip
    
    Пусть теперь $\bds a, \bds b \in \mathcal V_2$.
    При этом $\bds a \hm= (a_1, \ldots, a_n)$, $\bds b \hm= (b_1, \ldots, b_n)$.
    Тогда сумма векторов
    \[
      \bds a + \bds b = (a_1 + b_1, \ldots, a_n + b_n)
    \]
    
    И сумма координат у суммы:
    \[
      \sum_i (a_i + b_i) = \sum_i a_i + \sum_i b_i = 1 + 1 = 2 \not= 1
    \]
    
    $\mathcal V_2$ не замкнуто относительно суммы векторов (при умножении вектора на число тоже можно получить ``не то''), поэтому не является подпространством.
    А вообще ещё несложно было заметить, что в $\mathcal V_2$ даже нет нуля.
  \end{solution}
  
  
  \subsection{\# 20.18 (``правильная'')}
  
  Доказать, что четыре матрицы
  \begin{equation}\label{p-20-18-matrices}
    \left\{
      \begin{pmatrix}
        1 & -1\\
        1 & 1
      \end{pmatrix},
      \begin{pmatrix}
        2 & 5\\
        1 & 3
      \end{pmatrix},
      \begin{pmatrix}
        1 & 1\\
        0 & 1
      \end{pmatrix},
      \begin{pmatrix}
        3 & 4\\
        5 & 7
      \end{pmatrix}
    \right\}
  \end{equation}
  образуют базис в пространстве квадратных матриц порядка $2$.
  
  Далее найти компоненты матрицы
  \[
    \begin{pmatrix}
      5 & 14\\
      6 & 13
    \end{pmatrix}
  \]
  в этом базисе.
  
  \begin{solution}
    Покажем, что четыре матрицы (\ref{p-20-18-matrices}) в самом деле можно взять в качестве базиса.
    Для этого можно просто проверить, как по системе раскладывается нулевая матрица (если единственной решение~---~тривиальное, то система линейно независима).
    Итак, составляем линейную комбинацию матриц (\ref{p-20-18-matrices}) и приравниваем её нулевой:
    \[
      \alpha \begin{pmatrix}
        1 & -1\\
        1 & 1
      \end{pmatrix} + \beta \begin{pmatrix}
        2 & 5\\
        1 & 3
      \end{pmatrix} + \gamma \begin{pmatrix}
        1 & 1\\
        0 & 1
      \end{pmatrix} + \zeta \begin{pmatrix}
        3 & 4\\
        5 & 7
      \end{pmatrix} = 0
    \]
  \end{solution}
  
  Составляем систему:
  \[
    \left\{
      \begin{aligned}
        &\alpha + 2\beta + \gamma + 3\zeta = 0\\
        &\alpha + \beta \hphantom{+ \gamma} + 5\zeta = 0\\
        &-\alpha + 5\beta + \gamma + 4\zeta = 0\\
        &\alpha + 3\beta + \gamma + 7\zeta = 0
      \end{aligned}
    \right.
  \]
  
  И решаем её методом Гаусса (на каждом шаге цветом выделен элемент, который используем для ``устранения'' других ненулевых в том же столбце\footnote{И каждый раз такой элемент выбирается в новой строке.}):
  
  \begin{equation}\label{p-20-18-independence}
  \begin{split}
    \begin{pmatrix}
      \textcolor{pink}{\bds 1} & 2 & 1 & 3\\
      1 & 1 & 0 & 5\\
      -1 & 5 & 1 & 4\\
      1 & 3 & 1 & 7
    \end{pmatrix}
    &\sim \begin{pmatrix}
      1 & 2 & 1 & 3\\
      0 & \textcolor{pink}{\bds{-1}} & -1 & 2\\
      0 & 7 & 2 & 7\\
      0 & 1 & 0 & 4
    \end{pmatrix}
    \sim \begin{pmatrix}
      1 & 0 & -1 & 7\\
      0 & -1 & -1 & 2\\
      0 & 0 & -5 & 21\\
      0 & 0 & \textcolor{pink}{\bds{-1}} & 6
    \end{pmatrix}\\
    &\sim \begin{pmatrix}
      1 & 0 & 0 & 1\\
      0 & -1 & 0 & -4\\
      0 & 0 & 0 & -9\\
      0 & 0 & -1 & 6
    \end{pmatrix}
    \sim \begin{pmatrix}
      1 & 0 & 0 & 1\\
      0 & -1 & 0 & -4\\
      0 & 0 & 0 & \textcolor{pink}{\bds 1}\\
      0 & 0 & -1 & 6
    \end{pmatrix}
    \sim \begin{pmatrix}
      1 & 0 & 0 & 0\\
      0 & -1 & 0 & 0\\
      0 & 0 & 0 & 1\\
      0 & 0 & -1 & 0
    \end{pmatrix}
  \end{split}
  \end{equation}
  
  Таким образом, векторы-матрицы линейно независимы\footnote{
    Можно бы было доказывать линейную независимость не ``через матрицы'', а ``через вектора''.
    То есть каждую матрицу из (\ref{p-20-18-matrices}) можно бы было ``развернуть'' в столбец.
    Например, первой матрице $\left(\begin{smallmatrix} 1 & -1 \\ 1 & 1\end{smallmatrix}\right)$ можно бы было сопоставить столбец $\left(\begin{smallmatrix} 1 & -1 & 1 & 1\end{smallmatrix}\right)^T$ (и оставшимся матрицам~---~по такому же правилу).
    После этого, возможно, было бы немного ``приятнее'' составлять линейную комбинацию из векторов-столбцов и приравнивать её нулевому столбцу.
    В итоге же всё равно получилась бы система линейных уравнений.
    Линейная же зависимость матриц и ``развёрнутых столбцов'' равносильны, потому что множества матриц второго порядка и столбцов размера четыре (которые сопоставляются матрицам по конкретному правилу) изоморфны.
  }.
  Их четыре (в четырёхмерном пространстве), поэтому это~---~полная система.
  Поэтому их можно взять в качестве базиса.
  
  \bigskip
  
  Чтобы теперь разложить матрицу $\left(\begin{smallmatrix} 5 & 14\\ 6 & 13 \end{smallmatrix}\right)$ по базису, надо представить её в виде их линейной комбинации (решение точно будет существовать, притом единственное, так как система матриц (\ref{p-20-18-matrices})~---~базис):
  \[
    \alpha \begin{pmatrix}
      1 & -1\\
      1 & 1
    \end{pmatrix} + \beta \begin{pmatrix}
      2 & 5\\
      1 & 3
    \end{pmatrix} + \gamma \begin{pmatrix}
      1 & 1\\
      0 & 1
    \end{pmatrix} + \zeta \begin{pmatrix}
      3 & 4\\
      5 & 7
    \end{pmatrix} = \begin{pmatrix}
      5 & 14\\
      6 & 13
    \end{pmatrix}
  \]
  
  Расширенная матрица системы:
  \[
    \left(
      \begin{matrix}
        1 & 2 & 1 & 3\\
        1 & 1 & 0 & 5\\
        -1 & 5 & 1 & 4\\
        1 & 3 & 1 & 7
      \end{matrix}
      \BigMiddleFour
      \begin{matrix}
        5 \\ 6 \\ 14 \\ 13
      \end{matrix}
    \right)
  \]
  
  Далее с расширенной матрице можно провести те же элементарные преобразования строк, что и в случае (\ref{p-20-18-independence}).
  При этом можно работать только с последним столбцом (так как для матрицы уже ``всё сделано''):
  \begin{equation*}
  \begin{split}
    \begin{pmatrix} 5 \\ 6 \\ 14 \\ 13 \end{pmatrix}
    \sim \begin{pmatrix} 5 \\ 1 \\ 19 \\ 8 \end{pmatrix}
    \sim \begin{pmatrix} 7 \\ 1 \\ 26 \\ 9 \end{pmatrix}
    \sim \begin{pmatrix} -2 \\ -8 \\ -19 \\ 9 \end{pmatrix}
    \sim \begin{pmatrix} -2 \\ -8 \\ 19/9 \\ 9 \end{pmatrix}
    \sim \begin{pmatrix} -37/9 \\ 4/9 \\ 19/9 \\ -33/9 \end{pmatrix}
  \end{split}
  \end{equation*}
  
  Поэтому коэффициенты в разложении
  \[
    (\alpha, \beta,  \gamma, \zeta) = \left(-\frac{37}{9}, -\frac{4}{9}, \frac{33}{9}, \frac{19}{9}\right)
  \]
  
  
  \subsection{\# 20.22(3)}
  
  % https://tex.stackexchange.com/questions/28465/multiple-footnotes-at-one-point
  % https://tex.stackexchange.com/a/35060
  
  Найти размерность и базис подпространства\footnote{Снова ``абстрактные векторы'' отождествляются с их координатными столбцами в некотором базисе.}\textsuperscript{,}\footnote{Система $Ax \hm= 0$ в самом деле определяет подпространство: можно проверить замкнутость относительно сложения векторов и умножения вектора на число.}
  \[
    \left\{
      \begin{pmatrix}
        x_1 \\ x_2 \\ x_3
      \end{pmatrix} \in \RR^3
      \BigMiddleThree
      \begin{pmatrix}
        -3 & 1 & -2\\
        6 & -2 & 4\\
        -15 & 5 & -10
      \end{pmatrix}
      \begin{pmatrix}
        x_1 \\ x_2 \\ x_3
      \end{pmatrix}
      = 0
    \right\}
  \]
  
  \begin{solution}
    Приводим матрицу к упрощённому виду, выражаем базисные переменные через свободные:
    \[
      \begin{pmatrix}
        -3 & 1 & -2\\
        6 & -2 & 4\\
        -15 & 5 & -10
      \end{pmatrix}
      \sim \begin{pmatrix}
        -3 & 1 & -2\\
        0 & 0 & 0\\
        0 & 0 & 0
      \end{pmatrix}
      \leftrightarrow -3x_1 + x_2 - 2x_3 = 0
      \Leftrightarrow x_2 = 3x_1 + 2x_3
    \]
    
    Таким образом, решение системы:
    \[
      \begin{pmatrix}
        \alpha \\ 3\alpha + 2\gamma \\ \gamma
      \end{pmatrix}
      = \alpha \begin{pmatrix}
        1 \\ 3 \\ 0
      \end{pmatrix} + \gamma \begin{pmatrix}
        0 \\ 2 \\ 1
      \end{pmatrix}
    \]
    
    Вектора базиса найдены, их два, поэтому размерность подпространства равна двум.
  \end{solution}
\end{document}
