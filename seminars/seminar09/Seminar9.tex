\documentclass[a4paper,12pt]{article}

\usepackage{mystyle}

\graphicspath{ {images/} }

\definecolor{my-red}{RGB}{176, 0, 0}
\definecolor{my-blue}{RGB}{0, 0, 153}


\author{Алексеев Василий}


\title{Семинар 9}
\date{10 ноября + 16 ноября 2020}


\begin{document}
  \maketitle
  
  \tableofcontents

  \thispagestyle{empty}
  
  \newpage
  
  \pagenumbering{arabic}


  \section{Поверхности второго порядка}
  
  Любую поверхность второго порядка, как и кривую второго порядка на плоскости, можно задать в некоторой общей декартовой системе координат уравнением второй степени от координат точки.
  Правда, в случае поверхностей, помимо $x$ и $y$, добавляется ещё одна переменная~---~пусть это будет переменная $z$.
  
  Так же, как и для кривой второго порядка на плоскости, общее уравнение поверхности второго порядка с помощью ряда замен переменных можно привести к одному из нескольких канонических видов.  % TODO: check Bec
  Более того, некоторые поверхности второго порядка можно получить вращением вокруг оси симметрии соответствующей кривой второго порядка.
  Далее рассмотрим некоторые из кривых второго порядка.
  
  
  \subsection{Эллипсоид}
  
  Пусть на плоскости с выбранной прямоугольной декартовой системой координат $OXZ$ эллипс задан уравнением:
  \[
    \frac{x^2}{a^2} + \frac{z^2}{c^2} = 1
  \]
  
  Перейдём в пространство.
  Выберем в пространстве прямоугольную декартову систему координат $OXYZ$ (пусть ось $OY$ проведена так, чтобы система координат в пространстве была правой).
  И будем вращать указанный ранее эллипс вокруг оси $OX$:
  % TODO: pic
  Все точки эллипса будут вращаться по окружностям, ``нанизанным'' на ось $OX$.
  Рассмотрим точку $M_0$ эллипса.
  У точки $M_0$ координаты $(x_0, 0, z_0)$.
  При вращении она в какой-то момент перейдёт в точку $M'$ с координатами $(x, y, z_0)$.
  Расстояние до оси вращения как от точки $M_0$, так и от точки $M$, одинаково:
  \[
    \sqrt{x_0^2 + 0^2} = \sqrt{x^2 + y^2}
  \]
  
  При этом точка $M_0$ лежит на эллипсе:
  \[
    \frac{x_0^2}{a^2} + \frac{z_0^2}{c^2} = 1
  \]
  
  Поэтому для получения уравнения эллипсоида (от координат $x$, $y$, $z$) надо заменить $x_0^2$ в равенстве выше на $x^2 \hm+ y^2$ и $z_0^2$ на $z^2$:
  \[
    \frac{x^2 + y^2}{a^2} + \frac{z^2}{c^2} = 1
  \]
  
  Итак, каждая точка эллипса при вращении будет двигаться по окружности.
  Уравнению выше удовлетворяет любая точка любой такой окружности~---~траектории вращения точки эллипса.
  По построению, только из таких точек и состоит описанный эллипсоид.
  Поэтому полученной уравнение~---~уравнение эллипсоида.
  
  Если дополнительно провести сжатие вдоль оси $OY$, то можно прийти к общему уравнению эллипсоида:
  \begin{equation}
    \label{eq:ellipsoid}
    \frac{x^2}{a^2} + \frac{y^2}{b^2} + \frac{z^2}{c^2} = 1
  \end{equation}
  
  Если исходный эллипс вращать вокруг оси $OZ$, а не $OX$, то эллипсоид получится другой, но его уравнение всё равно будет вида (\ref{eq:ellipsoid}).  % TODO: будет вида?
  
  
  \subsection{Гиперболоид}
  
  
  \subsubsection{Однополостный}
  
  Аналогично получению уравнения эллипсоида, рассмотрим гиперболу в её канонической системе координат и будем вращать её вокруг оси симметрии (рассматривая уже пространство с прямоугольной системой координат, которая получена из канонической системы координат гиперболы).
  
  Уравнение гиперболы:
  \[
    \frac{x^2}{a^2} - \frac{z^2}{c^2} = 1
  \]
  
  Если вращать вокруг оси $OX$, то, по аналогии с эллипсом и эллипсоидом, получаем:
  % TODO: pic
  \[
    \frac{x^2 + y^2}{a^2} - \frac{z^2}{c^2} = 1
  \]
  
  И после сжатия вдоль оси $OY$:
  \begin{equation}
    \label{eq:hyperboloid1}
    \frac{x^2}{a^2} + \frac{y^2}{b^2} - \frac{z^2}{c^2} = 1
  \end{equation}
  
  Полученная поверхность, которую в некоторой декартовой системе координат можно описать уравнением вида (\ref{eq:hyperboloid1}), называется \emph{однополостным гиперболоидом} (``однополостным''~---~потому что одна полость посередине, ``гиперболоидом''~---~потому что получен вращением гиперболы).
  
  Но у гиперболы, как и у эллипса, две оси симметрии.
  И можно бы было вращать гиперболу вокруг оси $OZ$...
  
  
  \subsubsection{Двуполостный}
  
  % TODO: pic
  
  ...В таком случае уравнение поверхности вращения получилось бы таким:
  \[
    \frac{x^2}{a^2} - \frac{y^2 + z^2}{c^2} = 1
  \]
  
  И после сжатия, опять вдоль оси $OY$:  % TODO: про сжатие
  \begin{equation}
    \label{eq:hyperboloid2}
    \frac{x^2}{a^2} - \frac{y^2}{b^2} - \frac{z^2}{c^2} = 1
  \end{equation}
  
  Полученная поверхность вращения называется \emph{двуполостным гиперболоидом} (потому что уже две полости).
  В некоторой декартовой системе координат двуполостный гиперболоид описывается уравнением (\ref{eq:hyperboloid2}).
  
  
  \subsection{Параболоид}
  
  
  \subsubsection{Эллиптический}
  
  Перейдём в вращению параболы вокруг оси симметрии.
  Пусть парабола задана в канонической системе координат уравнением
  \[
    x^2 = 2pz
  \]
  
  При вращении вокруг оси $OX$ получим поверхность
  \[
    x^2 + y^2 = 2pz
  \]
  
  % TODO: pic
  
  Или, после сжатия-растяжения вдоль осей $OX$ и $OY$, можно прийти к уравнению вида:
  \begin{equation}
    \label{eq:paraboloid1}
    \frac{x^2}{a^2} + \frac{y^2}{b^2} = 2z
  \end{equation}
  
  Поверхность % TODO: ref pic
  называется \emph{эллиптическим параболоидом} (``эллиптическим''~---~потому что в сечении плоскостями вида $z \hm= C$ получаются эллипсы).
  
  В полученном уравнении (\ref{eq:paraboloid1}) можно поменять знак ``плюс'' на ``минус'', и тогда получится...
  
  
  \subsubsection{Гиперболический}
  
  ...следующее уравнение:
  \begin{equation}
    \label{eq:paraboloid2}
    \frac{x^2}{a^2} - \frac{y^2}{b^2} = 2z
  \end{equation}
  
  Поверхность, описываемая в некоторой декартовой системе координат уравнением (\ref{eq:paraboloid2}) называется \emph{гиперболическим параболоидом} (``гиперболическим''~---~потому что в сечении плоскостями вида $z \hm= C$ получаются гиперболы).
  
  Гиперболический параболоид~---~не поверхность вращения. % TODO: поверзность вращения
  Опустим анализ уравнения (\ref{eq:paraboloid2}) и скажем просто, что о гиперболическом параболоиде можно думать как о поверхности, полученной при движении вершины одной параболы по другой параболе, так что оси парабол параллельны, ветви их направлены в противоположные стороны, а сами параболы лежат во взаимно перпендикулярных плоскостях...  % TODO: взаимно перпендикулярных
  
  % TODO: pic
  
  \section{Задачи}
  
  
  \subsection{\# 10.3(7)}
  
  Определить тип поверхности при разных $\lambda$:
  \[
    x^2 + \lambda (y^2 + z^2) = \lambda
  \]
  
  \begin{solution}
    Рассмотрим случай $\lambda \hm= 0$:
    \[
      x^2 = 0 \leftrightarrow x = 0
    \]
    
    Получается плоскость.
    
    \medskip
    
    Пусть теперь $\lambda \hm> 0$.
    Поделим обе части исходного уравнения на $\lambda$:
    \[
      \frac{x^2}{\lambda} + y^2 + z^2 = 1
    \]
    
    Это эллипсоид.
    
    \medskip
    
    Пусть теперь $\lambda \hm< 0$.
    Снова можно поделить обе части уравнения на $\lambda$, только теперь ``смотреть'' на левую часть стоит по-другому:
    \[
      y^2 + z^2 - \frac{x^2}{-\lambda} = 1
    \]
    
    Это~---~гиперболоид (однополостный).
  \end{solution}
  
  
  \subsection{\# 10.38}
  
  Составить уравнение прямого кругового цилиндра, проходящего через точку $M(1, 1, 2)$, и ось которого задана системой уравнений $x \hm= 1 \hm+ t$, $y \hm= 2 \hm+ t$, $z \hm= 3 \hm+ t$, $t \hm\in \RR$.
  
  \begin{solution}
    % TODO: pic
    
    Очевидно, данных в задаче в самом деле достаточно для задания цилиндра.
    Первым шагом хотелось бы найти радиус цилиндра...
    
    Направляющий вектор прямой-оси цилиндра $\bds a$ и точка $P_0$ на оси цилиндра: $\bds a \hm= (1, 1, 1)$, $P_0(1, 2, 3)$.
    
    Из определения цилиндра следует, что в сечении кругового цилиндра плоскостями, параллельными основанию, будут получаться окружности.
    Каждой точке на поверхности цилиндра соответствует плоскость $\alpha$, перпендикулярная оси и при сечении цилиндра дающая окружность, на которой лежит эта точка.
    А каждой такой окружности соответствует точка на оси цилиндра.
    Найдём точку $P$ на оси цилиндра, соответствующую точке $M$, данной в условии.
    Зная координаты точки $P$, можно будет найти радиус основания цилиндра как расстояние $d$ между точками $P$ и $M$: $R \hm= d(P, M)$.
    
    Направляющий вектор оси $\bds a$ является и вектором нормали плоскостей, перпендикулярных оси цилиндра.
    Поэтому семейство плоскостей, перпендикулярных оси, задаётся уравнением
    \begin{equation}
      \label{eq:family-of-planes}
      x + y + z + D = 0,\quad D \in \RR
    \end{equation}
    
    Найдём, какой точке $t_0$ на оси соответствует плоскость с заданным свободным членом~$D$:
    \[
      (1 + t_0) + (2 + t_0) + (3 + t_0) + D = 0 \Leftrightarrow D = -6 - 3t_0 \Leftrightarrow t_0 = \frac{-6 - D}{3}
    \]
    
    Плоскость $\alpha$ для точки $M(1, 1, 2)$:
    \[
      1 + 1 + 2 + D = 0 \Leftrightarrow D = -4
    \]
    
    Поэтому точка $P$ на оси (и на той же плоскости, что и точка $M$):
    \[
      t_P = \frac{-6 - (-4)}{3} = -\frac{2}{3}
    \]
    \[
      \left\{
        \begin{aligned}
          &x_P = 1 + t_P = 1 - \frac{2}{3} = \frac{1}{3}\\
          &y_P = 2 + t_P = 2 - \frac{2}{3} = 1\frac{1}{3}\\
          &z_P = 3 + t_P = 3 - \frac{2}{3} = 2\frac{1}{3}
        \end{aligned}
      \right.
    \]
    
    Теперь можно найти радиус цилиндра:
    \[
      R = d(P, M) = \sqrt{(x_M - x_P)^2 + (y_M - y_P)^2 + (z_M - z_P)^2} = \ldots = \frac{\sqrt{6}}{3}
    \]
    
    Теперь рассмотрим произвольную точку $M'(x, y, z)$ на цилиндре.
    Ей, как и точке $M$, соответствует некоторая плоскость $\alpha'$ из семейства (\ref{eq:family-of-planes}) и точка $P'$ на оси цилиндра:
    \[
      \alpha'\colon x + y + z + D' = 0 \Leftrightarrow D' = -(x + y + z)
    \]
    \begin{equation}
    \begin{split}
      P' \in \alpha' &\Leftrightarrow x_{P'} + y_{P'} + z_{P'} + D' = 0\\
      &\Leftrightarrow (1 + t_{P'}) + (2 + t_{P'}) + (3 + t_{P'}) + D' = 0
      \Leftrightarrow t_{P'} = \frac{-6 - D'}{3}
    \end{split}
    \end{equation}
    
    Поэтому
    \begin{equation}
      \label{eq:t_P_prime}
      t_{P'} = \frac{-6 + x + y + z}{3}
    \end{equation}
    
    И снова выписываем выражение для расстояния от точки $M'$ (в этот раз произвольной на цилиндре) до соответствующей ей точки $P'$:
    \[
      d(M', P') = \sqrt{(x - x_{P'})^2 + (y - y_{P'})^2 + (z - z_{P'})^2} = R
    \]
    
    Подставляя вместо $x_{P'}$, $y_{P'}$, $z_{P'}$ их выражения через $t_{P'}$, а затем заменяя $t_{P'}$ его представлением через координаты $x, y, z$ точки $M'$ (\ref{eq:t_P_prime}), и вспоминая найденный ранее $R$, получаем уравнение цилиндра:
    \[
      x^2 + y^2 + z^2 - xy - xz - yz + 3x - 3z + 2 = 0
    \]
  \end{solution}
  
  
  \subsection{\# 10.26}
  
  Найти уравнение прямого кругового цилиндра радиуса $R$ с осью $\bds r \hm= \bds r_0 + \bds a t$, $t \hm\in \RR$.
  
  \begin{solution}
    % TODO: pic
    
    Для произвольной точки $\bds r$ цилиндра верно то, что вектор $\bds r \hm- \bds r_0$ равен произвольному сдвигу вдоль вектора $\bds a$ и сдвигу вдоль произвольного направления, перпендикулярного $\bds a$, на расстояние, равное $R$.
    Таким образом, для точек цилиндра постоянен модуль проекции вектора $\bds r \hm- \bds r_0$ на направление, перпендикулярное $\bds a$:
    \[
      |\bds r| \sin\angle{(\bds r, \bds a)} = R
    \]
    
    Полученное выражение и можно считать уравнением цилиндра.
    Его можно привести к более ``рабочему'' виду, если умножить обе части на $|\bds a|$:
    \[
      [\bds r, \bds a] = R |\bds a|
    \]
  \end{solution}
  
  
  \subsection{\# 10.41}
  
  Найти уравнение и определить тип поверхности, образованной вращением прямой $l$, заданной уравнениями $x \hm= 0$, $y \hm- z \hm+ 1 \hm= 0$, вокруг оси $OZ$.
  
  \begin{solution}
    Что может получаться при вращении одной прямой вокруг другой?
    Если прямые параллельны, то будет цилиндр.
    Если пересекаются~---~конус. % TODO: про конус
    Если скрещиваются... см. задачу 10.40.
    
    Проверим, как расположены друг относительно друга две данные в условии прямые.
    Но сначала найдём направляющий вектор $\bds a$ и начальную точку $\bds r_0$ вращаемой прямой $l$:
    \[
      \bds a = \left(\begin{vmatrix}0 & 0\\ 1 & -1\end{vmatrix}, \begin{vmatrix}0 & 1\\ -1 & 0\end{vmatrix}, \begin{vmatrix}1 & 0\\ 0 & 1\end{vmatrix}\right)
      = (0, 1, 1)
    \]
    \[
      \bds r_0 = (0, 0, 1)
    \]
    
    Поэтому прямую можно задать в виде системы скалярных параметрических уравнений так:
    \[
      \left\{
        \begin{aligned}
          &x = 0\\
          &y = t\\
          &z = 1 + t
        \end{aligned}
      \right.
    \]
    
    Направляющий вектор и начальная точка для оси $OZ$:
    \[
      \bds a_1 = (0, 0, 1),\ \bds r_1 = (0, 0, 1)
    \]
    
    Очевидно, что вращаемая прямая $l$ и ось вращения $OZ$ пересекаются в одной точке, $(0, 0, 1)$.
    Поэтому искомая поверхность вращения~---~конус.
    
    % TODO: pic
    Рассмотрим произвольную точку $M(x, y, z)$ на конусе.
    Этой точке соответствует точка $M_0(x_0, y_0, z_0)$, находящаяся на том же уровне относительно оси $OZ$, что и точка $M$, и лежащая на вращаемой прямой $l$.
    Расстояния от обеих точек до оси вращения одинаковы.
    Запишем же в виде формул перечисленные свойства:
    \[
      \left\{
        \begin{aligned}
          &\sqrt{(x - 0)^2 + (y - 0)^2 + (z - z)^2} = \sqrt{(x_0 - 0)^2 + (y_0 - 0)^2 + (z - z)^2}\\
          &z = z_0\\
          &x_0 = 0,\ y_0 = t_0,\ z_0 = 1 + t_0
        \end{aligned}
      \right.
    \]
    
    Откуда, исключая из первого уравнения ``нулевые'' координаты и оставляя только $x$, $y$ и $z$ некоторой точки $M$ цилиндра, получаем
    \[
      x^2 + y^2 - (z - 1)^2 = 0
    \]
  \end{solution}


  \subsection{\# 10.65(2)}
  
  Найти центр сечения эллипсоида
  $
    x^2 \hm+ 2y^2 \hm+ 4z^2 \hm= 40
  $
  плоскостью
  $
    x + y + z = 7
  $.
  
  \begin{solution}
    Выразим $x$ из уравнения плоскости через $y$ и $z$ (то есть ``перейдём'' в секущую плоскость):
    \[
      x = 7 - y - z
    \]
    и подставим в уравнение эллипсоида, чтобы получить уравнение сечения:
    \[
      (7 - y - z)^2 + 2y^2 + 4z^2 = 40
    \]
    
    После приведения подобных членов получаем:
    \[
      3y^2 + 2yz + 5z^2 - 14y - 14z + 9 = 0
    \]
    
    Это уравнение кривой второго порядка.
    Координаты центра (если он существует) можно найти из системы уравнений  % TODO: check theory
    \[
      \left\{
        \begin{aligned}
          &3y + z - 7 = 0\\
          &y + 5z - 7 = 0
        \end{aligned}
      \right.
    \]
    
    Определитель системы:
    \[
      \Delta = \begin{vmatrix} 3 & 1 \\ 1 & 5 \end{vmatrix} = 15 - 1 = 14 \not= 0
    \]
    
    Таким образом, центр существует.
    И его можно найти, например, с помощью метода Крамера:
    \[
      \left\{
        \begin{aligned}
          &y = \frac{35 - 7}{14} = 2\\
          &z = \frac{21 - 7}{14} = 1
        \end{aligned}
      \right.
    \]
    
    И первая компонента:
    \[
      x = 7 - 2 - 1 = 4
    \]
    
    Поэтому центр~---~точка $(4, 2, 1)$.
  \end{solution}
\end{document}
