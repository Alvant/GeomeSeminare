\documentclass[a4paper,12pt]{article}

\usepackage{mystyle}

\graphicspath{ {images/} }


% https://tex.stackexchange.com/questions/5461/is-it-possible-to-change-the-size-of-an-arrowhead-in-tikz-pgf
\usetikzlibrary{arrows.meta}

% https://tex.stackexchange.com/questions/261591/arrow-above-text-like-widehat
\usepackage{esvect}

% https://tex.stackexchange.com/questions/32501/how-to-get-a-good-divisible-by-symbol
\DeclareRobustCommand{\divby}{%
  \mathrel{\vbox{\baselineskip.65ex\lineskiplimit0pt\hbox{.}\hbox{.}\hbox{.}}}%
}


\definecolor{light-cyan}{RGB}{0, 204, 204}
\definecolor{light-purple}{RGB}{138, 43, 226}


\author{Алексеев Василий}
\title{Семинар 2}
\date{8 сентября 2020}


\begin{document}
  \maketitle
  
  \tableofcontents

  \thispagestyle{empty}
  
  \newpage
  
  \pagenumbering{arabic}


  \section{Вектора (-ы?)}
  
  Вектор~---~направленный отрезок (\ref{fig:vector}).
  Вектор можно обозначать одной строчной буквой, например $\bds a$, или двумя: началом и концом, например $\vv{AB}$.
  
  \begin{figure}[h]
    \centering
    
    \begin{tikzpicture}
      \draw[-{Latex[length=5mm, width=2mm]}, thick] (0,0) -- (5,5);
      \draw[ultra thick, fill] (0,0) circle[radius=0.05];
    \end{tikzpicture}
    
    \caption{Вектор характеризуется направлением и величиной.}
    \label{fig:vector}
  \end{figure}
  
  \begin{definition}[Коллинеарность]  % TODO: pic
    Два ненулевых вектора $\bds a$ и $\bds b$ называются \emph{коллинеарными}, если существует прямая, которой они параллельны.
    Коллинеарность обозначается $\bds a \hm\parallel \bds b$.
    Если при этом $\bds a$ и $\bds b$ направлены в одну сторону, то можно писать $\bds a \hm\upuparrows \bds b$,
    если в разные стороны~---~$\bds a \hm\updownarrows \bds b$.
    Нулевой вектор коллинеарен любому вектору.
  \end{definition}
  
  \begin{definition}[Компланарность]  % TODO: pic
    Три ненулевых вектора $\bds a$, $\bds b$ и $\bds c$ называются \emph{компланарными}, если существует плоскость, которой они параллельны.
    Три вектора, два из которых ненулевые, а третий нулевой, всегда компланарны.
  \end{definition}
  
  \begin{definition}[Равенство векторов]
    Будем считать два вектора $\bds a$ и $\bds b$ равными, если они
    \begin{itemize}
      \item равны по длине $|\bds a| = |\bds b|$
      \item коллинеарны $\bds a \parallel \bds b$
      \item одинаково направлены $\bds a \upuparrows \bds b$
    \end{itemize}
    Точка приложения при равенстве не учитывается\footnote{То есть получается, что можно нарисовать несколько несовпадающих, но равных векторов. Хотя в зависимости от конкретной задачи может быть важным различать векторы с разной точкой приложения. Например, в физике, при действии сил на тело.}.
  \end{definition}
  
  На множестве векторов определены следующие операции:
  \begin{itemize}
    \item Сложение векторов:
      \[
        \vv{AB} + \vv{BC} = \vv{AC}
      \]
    \item Умножение вектора $\bds a$ на число $\alpha \in \RR$.
      Результирующий вектор обозначается как $\alpha \bds a$ и определяется свойствами:
      \[
        \left\{
          \begin{aligned}
            &|\alpha \bds a| = |\alpha| \cdot |\bds a|\\
            &\alpha \bds a \parallel \bds a\\
            &\left\{
               \begin{aligned}
                 &\alpha \bds a \upuparrows \bds a, \alpha > 0\\
                 &\alpha \bds a \updownarrows \bds a, \alpha < 0
               \end{aligned}
             \right.
          \end{aligned}
        \right.
      \]
  \end{itemize}
  
  Множество векторов в $\RR^3$ с введёнными операциями сложения и умножения на число из $\RR$ образуют линейное пространство.
  Но рассмотрим векторы на одной прямой: сложение и умножение на число не выводят с прямой.
  То же самое с векторами на плоскости: сложение и умножение на число даёт вектор, также лежащий в той же плоскости.
  Таким образом, не только векторы из всего $\RR^3$ образуют линейное пространство, но и векторы, параллельные одной прямой, и векторы, параллельные одной плоскости.
  Множество векторов из одного нулевого вектора также образуют линейное пространство.
  Таким образом,
  \begin{itemize}
    \item нульмерное векторное пространство~---~нулевой вектор
    \item одномерное векторное пространство
      \[
        \{\bds v \in \RR^3 \mid \bds v \parallel l\}, \quad \mbox{$l$~---~прямая}
      \]
    \item двумерное векторное пространство
      \[
        \{\bds v \in \RR^3 \mid \bds v \parallel \alpha\}, \quad \mbox{$\alpha$~---~плоскость}
      \]
    \item трёхмерное векторное пространство~---~$\RR^3$
  \end{itemize}
  
  \begin{definition}
    Линейная комбинация векторов $\bds a_1, \ldots, \bds a_n$:
    \[
      \alpha_1 \bds a_1 + \ldots + \alpha_n \bds a_n, \quad \alpha_i \in \RR, 1 \leq i \leq n
    \]
    
    Нетривиальная линейная комбинация~---~когда хотя бы один их коэффициентов $\alpha_i$ отличен от нуля:
    $\sum\limits_{i=1}^n \alpha_i^2 \hm> 0$.
  \end{definition}
  
  \begin{definition}[Линейно зависимая система векторов]
    Система векторов $\bds a_1, \ldots, \bds a_n$ называется линейно зависимой, если существует их нетривиальная линейная комбинация, равная нулевому вектору:
    \[
      \left\{
        \begin{aligned}
          &\alpha_1 \bds a_1 + \ldots + \alpha_n \bds a_n = \bds 0\\
          &\alpha_1^2 + \ldots + \alpha_n^2 > 0
        \end{aligned}
      \right.
    \]
  \end{definition}
  
  \begin{example}
    Система из одного нулевого вектора линейно зависима.
  \end{example}
  
  \begin{theorem}
    Система из $k > 1$ вектора линейно зависима тогда и только тогда, когда один из векторов системы представим как линейная комбинация остальных.
  \end{theorem}
  
  \begin{proof}
    Пусть $\bds a_1, \ldots, \bds a_n$~---~линейно зависимы.
    Это значит, что
    \[
      \alpha_1 \bds a_1 + \ldots + \alpha_n \bds a_n = \bds 0
    \]
    и некоторый $\alpha_j \hm{\not=} 0$.
    Поэтому
    \[
      \alpha_j = \sum\limits_{\substack{1 \leq i \leq n\\i \not= j}} -\frac{\alpha_i}{\alpha_j} \bds a_i
    \]
    
    И наоборот, пусть некоторый $\bds a_j$ представим как линейная комбинация остальных векторов из набора с коэффициентами $\alpha_i'$:
    \[
      \bds a_j = \sum\limits_{\substack{1 \leq i \leq n\\i \not= j}} \alpha_i' \bds a_i
    \]
    Тогда
    \[
      \alpha_1' \bds a_1 + \ldots + (-1) \cdot \bds a_j + \ldots + \alpha_n' \bds a_n = \bds 0
    \]
    и по крайней мере один коэффициент $-1$ в линейной комбинации векторов $\{\bds a_i\}_{i=1}^n$ не равен нулю.
  \end{proof}
  
  \begin{theorem}\label{theo:linear-dependence-criteria}
    Критерии линейно зависимости систем векторов:
    \begin{itemize}
      \item Один вектор линейно зависим $\Leftrightarrow$ это нулевой вектор.
      \item Два вектора линейно зависимы $\Leftrightarrow$ эти векоры коллинеарны.
      \item Три ветора линейно зависимы $\Leftrightarrow$ эти векторы компланарны.
      \item Любые четыре вектора линейно зависимы\footnote{Мы в $\RR^3$.}.
    \end{itemize}
  \end{theorem}
  
  \begin{definition}[Базис]
    Базисом в пространстве называется
    \begin{itemize}
      \item упорядоченная
      \item линейно независимая
      \item полная\footnote{Любой вектор пространства модет быть разложен по системе.}
    \end{itemize}
    система векторов.
  \end{definition}
  
  Из теоремы (\ref{theo:linear-dependence-criteria}) следует, что
  \begin{itemize}
    \item В нулевом пространстве не существует базиса.
    \item В одномерном пространстве ненулевой вектор образует базис.
    \item В двумерном пространстве пара неколлинеарных векторов обращует базис.
    \item В трёхмерном пространстве тройка некомпланарных векторов образует базис.
  \end{itemize}
  \begin{problem}[1.6]
  
  \end{problem}
  
  \begin{solution}
  
  \end{solution}
  

  \begin{problem}[1.11(1)]
  
  \end{problem}
  
  \begin{solution}
  
  \end{solution}
  
  
  \begin{problem}[1.24(1)]
  
  \end{problem}
  
  \begin{solution}
  
  \end{solution}
  
  
  \begin{problem}[1.51]
  
  \end{problem}
  
  \begin{solution}
  
  \end{solution}
  
  
  \begin{problem}[1.39]
  
  \end{problem}
  
  \begin{solution}
  
  \end{solution}
  
  
  
  \begin{problem}[1.37]
  
  \end{problem}
  
  \begin{solution}
  
  \end{solution}
  
  
  
  \begin{problem}[1.36]
  
  \end{problem}
  
  \begin{solution}
  
  \end{solution}
  
  
  
  \section{Дополнение}
  
  \subsection{Про центр масс}
  
  
\end{document}