\documentclass[a4paper,12pt]{article}

\usepackage{mystyle}
\usepackage{gensymb}

\graphicspath{ {images/} }


% https://tex.stackexchange.com/questions/5461/is-it-possible-to-change-the-size-of-an-arrowhead-in-tikz-pgf
\usetikzlibrary{arrows.meta}

% https://tex.stackexchange.com/questions/261591/arrow-above-text-like-widehat
\usepackage{esvect}

% https://tex.stackexchange.com/questions/32501/how-to-get-a-good-divisible-by-symbol
\DeclareRobustCommand{\divby}{%
  \mathrel{\vbox{\baselineskip.65ex\lineskiplimit0pt\hbox{.}\hbox{.}\hbox{.}}}%
}


\definecolor{light-cyan}{RGB}{0, 204, 204}
\definecolor{light-purple}{RGB}{138, 43, 226}


\author{Алексеев Василий}
\title{Семинар 3}
\date{15 + 21 сентября 2020}


\begin{document}
  \maketitle
  
  \tableofcontents

  \thispagestyle{empty}
  
  \newpage
  
  \pagenumbering{arabic}


  \section{Замена базиса и системы координат}
  
  Будем обозначать векторы базиса в виде строки:
  \[
    \bds e \hm= (\bds e_1, \bds e_2, \bds e_3)
  \]
  для случая базиса в $\RR^3$.
  Аналогично и для базисов в $\RR^2$, $\RR$.
  
  При заданном базисе $\bds e$ любой вектор пространства $\bds x$ однозначно определяется его компонентами в базисе:
  \[
    \bds x = x_1 \cdot \bds e_1 + x_2 \cdot \bds e_2 + x_3 \cdot \bds e_3
      \Rightarrow \bds x \leftrightarrow (x_1, x_2, x_3)^T
  \]
  поэтому, говоря о векторе, часто имеют в виду его компоненты в базисе (то понятия вектора как направленного отрезка и вектора как столбца из чисел взаимозаменяемы).
  Но это при фиксированном базисе.
  
  В пространстве существует больше одного базиса: любая тройка некомпланарных векторов в $\RR^3$ образует базис.
  Встаёт вопрос о том, как связаны компоненты одного и того же вектора в разных базисах\footnote{В приложениях бывает нужно переводить координаты радиусов-векторов точек из одной системы координат в другую.}
  
  Пусть есть два базиса: $\bds e$ и $\bds e'$.  % TODO: pic with two basises
  Тогда векторы любого базиса можно разложить по системе векторов другого базиса.
  Разложим, например, векторы $\bds e$ по $\bds e'$:
  \begin{equation}\label{eq:e1-to-e2-via-system-of-equations}
    \left\{
      \begin{aligned}
        &\bds e_1 = a_{11} \cdot \bds e'_1 + a_{12} \cdot \bds e'_2 + a_{13} \cdot \bds e'_3\\
        &\bds e_2 = a_{21} \cdot \bds e'_1 + a_{22} \cdot \bds e'_2 + a_{23} \cdot \bds e'_3\\
        &\bds e_3 = a_{31} \cdot \bds e'_1 + a_{32} \cdot \bds e'_2 + a_{33} \cdot \bds e'_3
      \end{aligned}
    \right.
  \end{equation}
  
  Запись можно представить более компактно\footnote{Под результатом умножения строки из векторов $\bds e'$ на матрицу из чисел $S$ будем иметь в виду такую строку $\bds e$ из векторов, где каждый элемент равен линейной комбинации векторов умножаемой строки $\bds e'$ с коэффициентами, равными элементам соответственного столбца матрицы $S$. То есть по правилу умножения числовых матриц.}:
  \[
    \bds e = \bds e' S
  \]
  где $S$ называется \emph{матрицей перехода} от базиса $\bds e'$ к базису $\bds e$:
  \[
    S = \begin{pmatrix}
      a_{11} & a_{21} & a_{31}\\
      a_{12} & a_{22} & a_{32}\\
      a_{13} & a_{23} & a_{33}
    \end{pmatrix}
  \]
  во введённых ранее обозначениях (\ref{eq:e1-to-e2-via-system-of-equations}).
  
  Посмотрим теперь, как выражаются компоненты некоторого вектора $\bds x$ в одном базисе через его же компоненты, но в другом базисе.
  Имеем
  \begin{equation}\label{eq:same-vector-two-faces}
    \bds x = \bds e \bds x_e = \bds e' \bds x_{e'}
  \end{equation}
  где $\bds x$~---~вектор как направленный отрезок,
  $\bds x_e$~---~вектор-столбец, соответствующий $\bds x$ в базисе $\bds e$,
  $\bds x_{e'}$~---~вектор-столбец, соответствующий $\bds x'$ в базисе $\bds e'$.
  
  Теперь воспользуемся тем, что нам известно представление базиса $\bds e$ через вектора базиса $\bds e'$:
  \[
    \bds e \bds x_e = \bigl(\bds e' S\bigr) \bds x_e
      \stackrel{(\ref{eq:same-vector-two-faces})}{=} \bds e' \bds x_{e'}
  \]
  Так как умножение матриц ассоциативно, а также дистрибутивно относительно матричного сложения, мы можем перенести $\bds e' \bds x_{e'}$ влево и перегруппировать слагаемые:
  \[
    \bds e' \cdot \bigl(S \bds x_e - \bds x_{e'}\bigr) = \bds 0
  \]
  
  Из линейной независимости системы векторов $\bds e'$ получаем:
  \[
    S \bds x_e - \bds x_{e'} = \bds 0 \Leftrightarrow \bds x_{e'} = S \bds x_e
  \]
  
  Итак, в двух базисах компоненты векторов связаны так:
  \begin{equation}\label{eq:vector1-to-vector2}
    \boxed{
      \left\{
        \begin{aligned}
          &\bds e = \bds e' S\\
          &\bds x_{e'} = S \bds x_e
        \end{aligned}
      \right.
    }
  \end{equation}
  
  При этом, при переходе, наоборот, от базиса $\bds e$ к базису $\bds e'$
  можно написать аналогичное соотношение, но уже с другой матрицей перехода, которую обозначим за $S'$:
  \[
    \left\{
      \begin{aligned}
        &\bds e' = \bds e S'\\
        &\bds x_{e} = S' \bds x_{e'}
      \end{aligned}
    \right.
  \]
  
  Последний вопрос: как изменяются радиусы-векторы точек при смене системы координат?
  % TODO: pic
  Очевидно,
  \[
    \bds r_O(A) = \bds r_O(O') + \bds r_{O'}(A)
  \]
  где $\bds r_O(A)$~---~радиус-вектор точки $A$ в системе $O; \bds e$,
  $\bds r_{O'}(A)$~---~радиус-вектор точки $A$ в системе $O'; \bds e'$,
  и $\bds r_O(O')$~---~радиус-вектор, определяющий положение начала отсчёта $O'$ в системе $O; \bds e$.
  В системе $O'; \bds e'$ известно координатное представление вектора $ \bds r_{O'}(A)$.
  Для других же двух векторов $\bds r_O(A)$ и $\bds r_O(O')$ известны компоненты в базисе $O; \bds e$.
  Как записать соотношение выше через вектор-столбцы компонент векторов в базисах?
  Для этого надо все векторы представить в одном базисе.
  Из соотношения (\ref{eq:vector1-to-vector2}) мы можем выразить вектор $\bds r_{O'}(A)$ в базисе $\bds e$:
  \[
    \bds r_{O'}(A) = S' \bds x_{O'}(A)
  \]
  где $\bds x_{O'}(A)$~---~компоненты радиус-вектора точки $A$ в $O'; \bds e'$ и $S' \bds x_{O'}(A)$~---~компоненты \emph{того же} вектора в системе $O; \bds e$.
  Итого, получаем соотношение для компонент радиусов-векторов точки в разных системах координат:
  \begin{equation}\label{eq:point1-to-point2}
    \boxed{
      \left\{
        \begin{aligned}
          &\bds e' = \bds e S'\\
          &\bds x_O(A) = \bds x_O(O') + S' \bds x_{O'}(A)
        \end{aligned}
      \right.
    }
  \end{equation}
  то есть для нахождения координат точки в одной системе по её координатам в другой системе координат надо знать связь между векторами базисов и положение начала координат одной системы относительно другой.
  
  \begin{problem}[4.5]
    Есть две системы координат: $O; \bds e$ и $O'; \bds e'$.
    Координаты произвольной точки в первой системе обозначаются за $(x, y)$, координаты той же точки, но во второй системе координат~---~$(x', y')$.
    Известна связь между $(x, y)$ и $(x', y')$:
    \[
      \left\{
        \begin{aligned}
          &x = 2x' - y' + 5\\
          &y = 3x' + y' + 2
        \end{aligned}
      \right.
    \]
    
    Требуется найти
    \begin{itemize}
      \item Выражение $(x', y')$ через $(x, y)$.
      \item Координаты точки $O$ и компоненты векторов $\bds e_1, \bds e_2$ в системе $O'; \bds e'$.
      \item Координаты точки $O'$ и компоненты векторов $\bds e_1', \bds e_2'$ в системе $O; \bds e$.
    \end{itemize}
  \end{problem}
  
  \begin{solution}
    Перепишем связь между координатами точки в разных системах в матричном виде:
    \begin{equation}\label{eq:problem1-x-to-x-prime}
      \begin{pmatrix}
        x\\
        y
      \end{pmatrix}
      = \overbrace{\begin{pmatrix}
        2 & -1\\
        3 & 1
      \end{pmatrix}}^{S'}
      \begin{pmatrix}
        x'\\
        y'
      \end{pmatrix}
      + \begin{pmatrix}
        5\\
        2
      \end{pmatrix}
    \end{equation}
    
    Перепишем как
    \[
      \begin{pmatrix}
        2 & -1\\
        3 & 1
      \end{pmatrix}
      \begin{pmatrix}
        x'\\
        y'
      \end{pmatrix}
      = \begin{pmatrix}
        x - 5\\
        y - 2
      \end{pmatrix}
    \]
    
    И решим получивщуюся систему относительно $(x', y')$ с помощью метода Крамера:
    \[
      \Delta = \begin{vmatrix} 2 & -1 \\ 3 & 1 \end{vmatrix} = 2 + 3 = 5 \not= 0
    \]
    \[
      \Delta_{x'} = \begin{vmatrix} x - 5 & -1 \\ y - 2 & 1 \end{vmatrix} = x + y - 7
    \]
    \[
      \Delta_{y'} = \begin{vmatrix} 2 & x - 5 \\ 3 & y - 2 \end{vmatrix} = -3x + 2y + 11
    \]
    
    И сами координаты:
    \[
      x' = \frac{\Delta_{x'}}{\Delta} = \frac{1}{5} x + \frac{1}{5} y - \frac{7}{5}\\
    \]
    \[
      y' = \frac{\Delta_{y'}}{\Delta} = -\frac{3}{5} x + \frac{2}{5} y + \frac{11}{5}
    \]
    
    Или в матричном виде:
    \begin{equation}\label{eq:problem1-x-prime-to-x}
      \begin{pmatrix}
        x'\\
        y'
      \end{pmatrix}
      = \overbrace{\begin{pmatrix}
        1/5 & 1/5\\
        -3/5 & 2/5
      \end{pmatrix}}^{S}
      \begin{pmatrix}
        x\\
        y
      \end{pmatrix}
      + \begin{pmatrix}
        -7/5\\
        11/5
      \end{pmatrix}
    \end{equation}
    
    % TODO: про S' = S^-1
    
    Вспоминая (\ref{eq:point1-to-point2}) или просто подставляя нулевые векторы в соотношения для координат, получаем положения начал отсчёта:
    \begin{itemize}
      \item Нулевой вектор в (\ref{eq:problem1-x-to-x-prime}) $\Rightarrow \bds x_O(O') = (5, 2)^T$
      \item Нулевой вектор в (\ref{eq:problem1-x-prime-to-x}) $\Rightarrow \bds x_{O'}(O) = \left(-\dfrac{7}{5}, \dfrac{11}{5}\right)^T$
    \end{itemize}
    
    Вспоминая, что столбцы матриц $S$ и $S'$ есть компоненты векторов одного базиса в другом, или просто умножая матрицы $S$ и $S'$ на векторы $(1, 0)^T$ и $(0, 1)^T$, получаем компоненты одних базисных векторов в другом базисе:
    \begin{itemize}
      \item Столбцы $S$ ($\bds e \hm= \bds e' S$)
        \[
          \Rightarrow \left\{
            \begin{aligned}
              &\bds e_1 = \frac{1}{5} \bds e_1' - \frac{3}{5} \bds e_2'\\
              &\bds e_2 = \frac{1}{5} \bds e_1' + \frac{2}{5} \bds e_2'
            \end{aligned}
          \right.
        \]
      \item Столбцы $S'$ ($\bds e' \hm= \bds e S'$)
        \[
          \Rightarrow \left\{
            \begin{aligned}
              &\bds e_1' = 2 \bds e_1 + 3 \bds e_2\\
              &\bds e_2' = -\bds e_1 + \bds e_2
            \end{aligned}
          \right.
        \]
    \end{itemize}
  \end{solution}
  
  
  \begin{problem}[4.19]
    Треугольная призма $A B C A_1 B_1 C_1$.
    Точка $M$~---~точка пересечения медиан грани $A_1 B_1 C_1$.
    Требуется, зная координаты точки $x', y', z')$ в системе $A_1; \vv{A_1 B}, \vv{A_1 C}, \vv{A_1 M}$, найти её координаты $(x, y, z)$ в системе $A; \vv{AB}, \vv{AC}, \vv{AB_1}$.
  \end{problem}
  
  \begin{solution}
    % TODO: pic
    % TODO: note про порядок базисный векторов
    
    Что нам надо найти?
    Вспоминая формулы (\ref{eq:vector1-to-vector2}) или (\ref{eq:point1-to-point2}), получаем, что если векторы базиса связаны соотношением $\bds e' \hm= \bds e S'$, то компоненты векторов связаны соотношением $\bds x \hm= S' \bds x'$ и координаты точек связаны соотношением $\bds x_{O} \hm= \bds x_{O}(O') \hm+ S' \bds x_{O'}$.
    Таким образом, чтобы решить задачу, надо найти матрицу $S'$, столбцы которой~---~компоненты базиса $\vv{A_1 B}, \vv{A_1 C}, \vv{A_1 M}$ в $\vv{AB}, \vv{AC}, \vv{AB_1}$ и координаты начала отсчёта $A_1$ в системе с началом отсчёта $A$.
    
    Базисные векторы упорядочены.
    Разложим их по порядку:
    \[
      \vv{A_1 B} = \vv{A_1 A} + \vv{AB} = \vv{A_1 B_1} + \vv{B_1 A} + \vv{AB} = 2 \bds e_1 - \bds e_3  % TODO: ввести e_i
    \]
    \[
      \vv{A_1 C} = \vv{A_1 A} + \vv{AC} = \vv{A_1 B_1} + \vv{B_1 A} + \vv{AC} = \bds e_1 + \bds e_2 - \bds e_3
    \]
    \[
      \vv{A_1 M} = \frac{1}{3} (\vv{A_1 A_1} + \vv{A_1 B_1} + \vv{A_1 C_1}) = \frac{1}{3} (\vv{AB} + \vv{AC})
        = \frac{1}{3} (\bds e_1 + \bds e_2)
    \]
    
    Итого,
    \[
      (\bds e_1', \bds e_2', \bds e_3') = (\bds e_1, \bds e_2, \bds e_3) \begin{pmatrix}
        2 & 1 & 1/3\\
        0 & 1 & 1/3\\
        -1 & -1 & 0
      \end{pmatrix}
    \]
    
    Положение $A_1$ в системе $A; \bds e$:
    \[
      \vv{AA_1} = \vv{AB_1} + \vv{B_1 A_1} = -\bds e_1 + \bds e_3
    \]
    
    Поэтому связь между координатами точек в разных системах:
    \[
      \bds x = \begin{pmatrix}
        2 & 1 & 1/3\\
        0 & 1 & 1/3\\
        -1 & -1 & 0
      \end{pmatrix} \bds x'
      + \begin{pmatrix}
        -1\\
        0\\
        1
      \end{pmatrix}
    \]
  \end{solution}
  
  
  \section{Скалярное произведение}
  
  \begin{definition}
    Скалярное произведение $(\bds a, \bds b)$ ненулевых векторов $\bds a$ и $\bds b$ определяется следующим образом:
    \begin{equation}
      (\bds a, \bds b) \equiv |\bds a| \cdot |\bds b| \cdot \cos \phi
    \end{equation}
    где $|\bds a|$ и $|\bds b|$~---~модули векторов $\bds a$ и $\bds b$,
    а $\phi$~---~угол между векторами $\bds a$ и $\bds b$ (не превосходящий $\pi$).  % TODO: about angle
    В случае, если хотя бы один из пары векторов нулевой, скалярное произведение этих векторов полагается равным нулю.
  \end{definition}
  
  Отметим несколько свойств скалярного произведения:
  \begin{itemize}
    \item $(\bds a, \bds b) = (\bds b, \bds a)$~---~симметричность
    \item $(\bds a, \bds a) = |\bds a|^2$~---~скалярный квадрат вектора равен квадрату его длины
    \item О равенстве нулю скаларного произведения:
      \[
        (\bds a, \bds b) = 0 \Leftrightarrow \bds a = 0\ \mbox{или}\ \bds b = 0\ \mbox{или}\ \bds a \perp \bds b
      \]
    \item Линейность по первому аргументу:
      \[
        (\alpha \bds a + \beta \bds b, \bds c) = \alpha (\bds a, \bds c) + \beta (\bds b, \bds c)
      \]
  \end{itemize}
  
  Первые три свойства следуют из определения.
  Докажем последнее свойство.
  Начнём с того, что при заданном направлении $\bds l$ любой вектор раскладывается в сумму двух:
  \[
    \bds r = \bds r_{\parallel} + \bds r_{\perp}
  \]
  где $\bds r_{\parallel}$~---~вектор, параллельный $\bds l$, и $\bds r_{\perp}$~---~вектор, перпендикулярный $\bds l$.
  Компонента $\bds r_{\parallel}$ называется \emph{ортогональной векторной проекцией} вектора $\bds r$ на направление, определяемое вектором $\bds l$, и может обозначаться так:
  \[
    \pi_{\bds l}(\bds r) = \bds r_{\parallel}
  \]
  % TODO: scalar vs. vector proj
  
  % TODO: pic
  Спроецируем теперь вектор $\alpha \bds a \hm+ \beta \bds b$ на направление, определяемое вектором $\bds c$:
  \[
    \pi_{\bds c}(\alpha \bds a + \beta \bds b) = |\alpha \bds a + \beta \bds b| \cdot \cos \phi
  \]
  где $\pi_{\bds c}(\cdot)$~---~скалярная проекция на направление вектора $\bds c$, % TODO: footnote
  $\phi$~---~угол между вектором $\alpha \bds a \hm+ \beta \bds b$ и вектором $\bds c$.
  Но проекция вектора, являющегося суммой нескольких векторов, равна сумме проекций этих векторов: % TODO: footnote
  \[
    \pi_{\bds c}(\alpha \bds a + \beta \bds b) = \pi_{\bds c}(\alpha \bds a) + \pi_{\bds c}(\beta \bds b)
  \]
  поэтому
  \[
    |\alpha \bds a + \beta \bds b| \cdot \cos \phi = |\alpha \bds a| \cdot \cos \phi_1 + |\beta \bds b| \cdot \cos \phi_2
  \]
  где $\phi_1$ и $\phi_2$~---~углы, которые образуют векторы $\alpha \bds a$ и $\beta \bds b$ с вектором $\bds c$.
  Умножая обе части последнего равенства на модуль вектора $\bds c$, получаем то, что хотели доказать (при этом числовые множители можно вынести за знак модуля):
  \[
    (\alpha \bds a + \beta \bds b, \bds c) = \alpha (\bds a, \bds c) + \beta (\bds b, \bds c)
  \]
  \qed
  
  
  \begin{problem}[2.21]
    Длины базисных векторов $\bds e_1, \bds e_2, \bds e_3$ равны соответственно $3, \sqrt{2}$ и $4$.
    Углы между векторами $\angle (\bds e_1, \bds e_2) \hm= \angle (\bds e_2, \bds e_3) \hm= 45\degree$, $\angle (\bds e_1, \bds e_3) \hm= 60\degree$.
    
    Надо найти длины сторон и углы параллелограмма, построенного на векторах с координатами $(1, -3, 0)$ и $(-1, 2, 1)$ в указанном базисе.
  \end{problem}
  
  \begin{solution}
    Обозначим данные нам векторы за $\bds a$ и $\bds b$:
    \[
      \left\{
        \begin{aligned}
          &\bds a = (1, -3, 0)\\
          &\bds b = (-1, 2, 1)
        \end{aligned}
      \right.
    \]
    
    Базис не ортонормированный, поэтому скалярные произведения надо будет считать ``по-честному''.
    
    Модули вектора $\bds a$:
    \[
      |\bds a| = \sqrt{(\bds a, \bds a)} = \sqrt{(\bds e_1 - 3\bds e_2)(\bds e_1 - 3\bds e_2)}
        = \sqrt{(\bds e_1, \bds e_1) - 6(\bds e_1, \bds e_2) + 9(\bds e_2, \bds e_2)}
        = \sqrt{9 - 18 + 18} = 3
    \]
    
    Аналогично для вектора $\bds b$:
    \[
      |\bds b| = \sqrt{(\bds b, \bds b)} = \sqrt{(-\bds e_1 + 2\bds e_2 + \bds e_3)(-\bds e_1 + 2\bds e_2 + \bds e_3)}
        = \ldots = 5
    \]
    
    Косинус угла между векторами $\bds a$ и $\bds b$:
    \[
      \cos\angle(\bds a, \bds b) = \frac{(\bds a, \bds b)}{|\bds a| \cdot |\bds b|}
        = \frac{(\bds e_1 - 3\bds e_2) \cdot (-\bds e_1 + 2\bds e_2 + \bds e_3)}{3 \cdot 5}
        = \ldots = -\frac{12}{15} = -\frac{4}{5}
    \]
    
    И острый угол параллелограмма можно найти как $\arccos{\left(\dfrac{4}{5}\right)}$.
  \end{solution}
  
  В случае же \textbf{ортонормированного} базиса формулы с применением скалярных произведений упрощаются:
  \[
    (\bds a, \bds b) = \sum_{i=1}^n a_i b_i
  \]
  \[
    |\bds a| = \sqrt{\sum_{i=1}^n a_i^2}
  \]
  \[
    \cos\angle(\bds a, \bds b) = \frac{\sum_{i=1}^n a_i b_i}{\sqrt{\sum_{i=1}^n a_i^2} \sqrt{\sum_{i=1}^n b_i^2}}
  \]
  
  
  \begin{problem}[2.24]
    Даны два вектора $\bds a$ и $\bds b$, причём $\bds a \hm{\not=} \bds 0$.
    Чему равна ортогональная проекция $\bds b$ на направление, определяемое вектором $\bds a$?
  \end{problem}
  
  \begin{solution}
    \[
      \pi_{\bds a}(\bds b) = |\bds b| \cos\angle(\bds b, \bds a) \cdot \frac{\bds a}{|\bds a|}
    \]
    где левый множитель есть скалярная проекция вектора $\bds b$ на направление $\bds a$,
    а правый~---~единичный вектор в направлении $\bds a$.
    Выражение можно записать по-другому, если домножить числитель и знаменатель на $|\bds a|$:
    \[
      \pi_{\bds a}(\bds b) = \frac{(\bds a, \bds b)}{|\bds a|^2} \bds a
    \]
    % TODO: про направление векторной проекции
  \end{solution}
\end{document}